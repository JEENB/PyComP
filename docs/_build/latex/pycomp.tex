%% Generated by Sphinx.
\def\sphinxdocclass{report}
\documentclass[letterpaper,10pt,english]{sphinxmanual}
\ifdefined\pdfpxdimen
   \let\sphinxpxdimen\pdfpxdimen\else\newdimen\sphinxpxdimen
\fi \sphinxpxdimen=.75bp\relax
\ifdefined\pdfimageresolution
    \pdfimageresolution= \numexpr \dimexpr1in\relax/\sphinxpxdimen\relax
\fi
%% let collapsible pdf bookmarks panel have high depth per default
\PassOptionsToPackage{bookmarksdepth=5}{hyperref}

\PassOptionsToPackage{booktabs}{sphinx}
\PassOptionsToPackage{colorrows}{sphinx}

\PassOptionsToPackage{warn}{textcomp}
\usepackage[utf8]{inputenc}
\ifdefined\DeclareUnicodeCharacter
% support both utf8 and utf8x syntaxes
  \ifdefined\DeclareUnicodeCharacterAsOptional
    \def\sphinxDUC#1{\DeclareUnicodeCharacter{"#1}}
  \else
    \let\sphinxDUC\DeclareUnicodeCharacter
  \fi
  \sphinxDUC{00A0}{\nobreakspace}
  \sphinxDUC{2500}{\sphinxunichar{2500}}
  \sphinxDUC{2502}{\sphinxunichar{2502}}
  \sphinxDUC{2514}{\sphinxunichar{2514}}
  \sphinxDUC{251C}{\sphinxunichar{251C}}
  \sphinxDUC{2572}{\textbackslash}
\fi
\usepackage{cmap}
\usepackage[T1]{fontenc}
\usepackage{amsmath,amssymb,amstext}
\usepackage{babel}



\usepackage{tgtermes}
\usepackage{tgheros}
\renewcommand{\ttdefault}{txtt}



\usepackage[Bjarne]{fncychap}
\usepackage{sphinx}

\fvset{fontsize=auto}
\usepackage{geometry}


% Include hyperref last.
\usepackage{hyperref}
% Fix anchor placement for figures with captions.
\usepackage{hypcap}% it must be loaded after hyperref.
% Set up styles of URL: it should be placed after hyperref.
\urlstyle{same}

\addto\captionsenglish{\renewcommand{\contentsname}{Contents:}}

\usepackage{sphinxmessages}
\setcounter{tocdepth}{1}



\title{PyComP}
\date{Mar 16, 2023}
\release{1.0}
\author{Jenish Raj Bajracharya}
\newcommand{\sphinxlogo}{\vbox{}}
\renewcommand{\releasename}{Release}
\makeindex
\begin{document}

\ifdefined\shorthandoff
  \ifnum\catcode`\=\string=\active\shorthandoff{=}\fi
  \ifnum\catcode`\"=\active\shorthandoff{"}\fi
\fi

\pagestyle{empty}
\sphinxmaketitle
\pagestyle{plain}
\sphinxtableofcontents
\pagestyle{normal}
\phantomsection\label{\detokenize{index::doc}}


\sphinxAtStartPar
PyComP is a Python library for compressing and decompressing data. It provides a simple and efficient way to reduce the size of data files without losing any information.

\sphinxAtStartPar
\sphinxstylestrong{Features}

\sphinxAtStartPar
PyComP has a range of features that make it a powerful tool for data compression:
\begin{itemize}
\item {} 
\sphinxAtStartPar
\sphinxstylestrong{Supports multiple algorithms:} PyComP supports several compression algorithms, including Huffman, Arithmetic, Range, ABS and ANS, which can be selected based on your specific requirements.

\item {} 
\sphinxAtStartPar
\sphinxstylestrong{Customizable compression level:} PyComP allows you to specify the compression level, which determines the balance between compression ratio and speed. Higher levels result in smaller file sizes but slower processing times.

\item {} 
\sphinxAtStartPar
\sphinxstylestrong{Easy\sphinxhyphen{}to\sphinxhyphen{}use functions:} PyComP provides a range of convenient functions for compressing and decompressing data and files.

\end{itemize}

\sphinxAtStartPar
\sphinxstylestrong{Compression algorithms}

\sphinxAtStartPar
Here is a list of algorithms implemented.
\begin{itemize}
\item {} 
\sphinxAtStartPar
\sphinxtitleref{Huffman codes \textless{}https://github.com/JEENB/PyComP/blob/version1.0/PyComP/compressors/huffman.py\textgreater{}}

\item {} 
\sphinxAtStartPar
\sphinxtitleref{rANS \textless{}https://github.com/JEENB/PyComP/blob/version1.0/PyComP/compressors/rANS.py\textgreater{}}

\item {} 
\sphinxAtStartPar
\sphinxtitleref{sANS \textless{}https://github.com/JEENB/PyComP/blob/version1.0/PyComP/compressors/sANS.py\textgreater{}}

\item {} 
\sphinxAtStartPar
\sphinxtitleref{uABS \textless{}https://github.com/JEENB/PyComP/blob/version1.0/PyComP/compressors/uABS.py\textgreater{}}

\item {} 
\sphinxAtStartPar
\sphinxtitleref{Arithmetic coder \textless{}https://github.com/JEENB/PyComP/blob/version1.0/PyComP/compressors/arithmetic.py\textgreater{}}

\item {} 
\sphinxAtStartPar
\sphinxtitleref{Symmetric Numeral \textless{}https://github.com/JEENB/PyComP/blob/version1.0/PyComP/compressors/symmetric\_numeral.py\textgreater{}}

\end{itemize}

\sphinxAtStartPar
\sphinxstylestrong{Install the \textasciigrave{}PyComP\textasciigrave{} package}
\begin{quote}

\sphinxAtStartPar
\sphinxcode{\sphinxupquote{git clone \textless{}repo\textgreater{} and cd}}

\sphinxAtStartPar
\sphinxcode{\sphinxupquote{pip install \sphinxhyphen{}e . \#install the package in a editable mode}}
\end{quote}

\sphinxstepscope


\chapter{PyComP}
\label{\detokenize{modules:pycomp}}\label{\detokenize{modules::doc}}
\sphinxstepscope


\section{Core}
\label{\detokenize{core:core}}\label{\detokenize{core::doc}}

\subsection{core.data module}
\label{\detokenize{core:module-core.data}}\label{\detokenize{core:core-data-module}}\index{module@\spxentry{module}!core.data@\spxentry{core.data}}\index{core.data@\spxentry{core.data}!module@\spxentry{module}}\index{Data (class in core.data)@\spxentry{Data}\spxextra{class in core.data}}

\begin{fulllineitems}
\phantomsection\label{\detokenize{core:core.data.Data}}
\pysigstartsignatures
\pysiglinewithargsret{\sphinxbfcode{\sphinxupquote{class\DUrole{w}{  }}}\sphinxcode{\sphinxupquote{core.data.}}\sphinxbfcode{\sphinxupquote{Data}}}{\emph{\DUrole{n}{symbols}\DUrole{p}{:}\DUrole{w}{  }\DUrole{n}{list}}, \emph{\DUrole{n}{frequency}\DUrole{p}{:}\DUrole{w}{  }\DUrole{n}{list}}}{}
\pysigstopsignatures
\sphinxAtStartPar
Bases: \sphinxcode{\sphinxupquote{object}}

\sphinxAtStartPar
This is the main class. Every compression algorithm inherits this class.
\begin{quote}\begin{description}
\sphinxlineitem{Parameters}\begin{itemize}
\item {} 
\sphinxAtStartPar
\sphinxstyleliteralstrong{\sphinxupquote{symbols}} \textendash{} list of symbols, elements can be any format

\item {} 
\sphinxAtStartPar
\sphinxstyleliteralstrong{\sphinxupquote{frequency}} \textendash{} frequency list associated with the list

\end{itemize}

\end{description}\end{quote}

\sphinxAtStartPar
: type freqeuncy: list


\subsubsection{Methods}
\label{\detokenize{core:methods}}\begin{description}
\sphinxlineitem{\_\_frequency\_distribution()}
\sphinxAtStartPar
Computes the frequency distribution and created attributes freq\_dist, M(sum of frequency)

\sphinxlineitem{\_\_cumul\_distribution()}
\sphinxAtStartPar
Computes the cumulative distribution of a symbol
Creates an attribute cum\_dist: dictionary with range
\begin{quote}
\begin{description}
\sphinxlineitem{\{}
\sphinxAtStartPar
s\_1:{[}low, high{]}

\end{description}

\sphinxAtStartPar
\}
\end{quote}

\end{description}

\begin{sphinxadmonition}{note}{Note:}
\sphinxAtStartPar
\_\_freqency\_distribuiton and \_\_cumul\_distribution are initialized
\end{sphinxadmonition}


\subsubsection{Raises}
\label{\detokenize{core:raises}}\begin{description}
\sphinxlineitem{ValueError}
\sphinxAtStartPar
If frequency and symbol list do not match.

\end{description}
\index{shannon\_entropy() (core.data.Data method)@\spxentry{shannon\_entropy()}\spxextra{core.data.Data method}}

\begin{fulllineitems}
\phantomsection\label{\detokenize{core:core.data.Data.shannon_entropy}}
\pysigstartsignatures
\pysiglinewithargsret{\sphinxbfcode{\sphinxupquote{shannon\_entropy}}}{\emph{\DUrole{n}{show\_steps}\DUrole{o}{=}\DUrole{default_value}{False}}}{{ $\rightarrow$ float}}
\pysigstopsignatures
\sphinxAtStartPar
Computes the shanon entroy as sum(p(x)log(p(x)))
\begin{description}
\sphinxlineitem{Parameters:}\begin{description}
\sphinxlineitem{show\_steps: bool, default = False}
\sphinxAtStartPar
Show the steps if bool is true

\end{description}

\sphinxlineitem{Returns:}
\sphinxAtStartPar
entropy: float
the entropy value

\end{description}

\begin{sphinxVerbatim}[commandchars=\\\{\}]
\PYG{g+gp}{\PYGZgt{}\PYGZgt{}\PYGZgt{} }\PYG{n}{symbols} \PYG{o}{=} \PYG{p}{[}\PYG{l+s+s1}{\PYGZsq{}}\PYG{l+s+s1}{a}\PYG{l+s+s1}{\PYGZsq{}}\PYG{p}{,} \PYG{l+s+s1}{\PYGZsq{}}\PYG{l+s+s1}{b}\PYG{l+s+s1}{\PYGZsq{}}\PYG{p}{,} \PYG{l+s+s1}{\PYGZsq{}}\PYG{l+s+s1}{c}\PYG{l+s+s1}{\PYGZsq{}}\PYG{p}{,} \PYG{l+s+s1}{\PYGZsq{}}\PYG{l+s+s1}{e}\PYG{l+s+s1}{\PYGZsq{}}\PYG{p}{]}
\PYG{g+gp}{\PYGZgt{}\PYGZgt{}\PYGZgt{} }\PYG{n}{frequency} \PYG{o}{=} \PYG{p}{[}\PYG{l+m+mi}{3}\PYG{p}{,} \PYG{l+m+mi}{4}\PYG{p}{,} \PYG{l+m+mi}{5}\PYG{p}{,} \PYG{l+m+mi}{1}\PYG{p}{]}
\PYG{g+gp}{\PYGZgt{}\PYGZgt{}\PYGZgt{} }\PYG{n}{d} \PYG{o}{=} \PYG{n}{Data}\PYG{p}{(}\PYG{n}{symbols}\PYG{p}{,} \PYG{n}{frequency}\PYG{p}{)}
\PYG{g+gp}{\PYGZgt{}\PYGZgt{}\PYGZgt{} }\PYG{n+nb}{print}\PYG{p}{(}\PYG{n}{d}\PYG{o}{.}\PYG{n}{shannon\PYGZus{}entropy}\PYG{p}{(}\PYG{p}{)}\PYG{p}{)}
\PYG{g+go}{    1.8262452584026092}
\end{sphinxVerbatim}

\sphinxAtStartPar
TODO: Implement show steps

\end{fulllineitems}


\end{fulllineitems}



\subsection{Module contents}
\label{\detokenize{core:module-core}}\label{\detokenize{core:module-contents}}\index{module@\spxentry{module}!core@\spxentry{core}}\index{core@\spxentry{core}!module@\spxentry{module}}
\sphinxstepscope


\section{Huffman Coding}
\label{\detokenize{huffman:module-huffman}}\label{\detokenize{huffman:huffman-coding}}\label{\detokenize{huffman::doc}}\index{module@\spxentry{module}!huffman@\spxentry{huffman}}\index{huffman@\spxentry{huffman}!module@\spxentry{module}}\index{Huffman (class in huffman)@\spxentry{Huffman}\spxextra{class in huffman}}

\begin{fulllineitems}
\phantomsection\label{\detokenize{huffman:huffman.Huffman}}
\pysigstartsignatures
\pysiglinewithargsret{\sphinxbfcode{\sphinxupquote{class\DUrole{w}{  }}}\sphinxcode{\sphinxupquote{huffman.}}\sphinxbfcode{\sphinxupquote{Huffman}}}{\emph{\DUrole{n}{symbols}\DUrole{p}{:}\DUrole{w}{  }\DUrole{n}{list}}, \emph{\DUrole{n}{frequency}\DUrole{p}{:}\DUrole{w}{  }\DUrole{n}{list}}}{}
\pysigstopsignatures
\sphinxAtStartPar
Bases: {\hyperref[\detokenize{core:core.data.Data}]{\sphinxcrossref{\sphinxcode{\sphinxupquote{Data}}}}}

\sphinxAtStartPar
Class for Huffman Encoding. This class inherits the data class.
\begin{description}
\sphinxlineitem{Attributes:}\begin{description}
\sphinxlineitem{symbols}{[}list{]}
\sphinxAtStartPar
list of symbols, elements can be any format

\sphinxlineitem{frequency: list}
\sphinxAtStartPar
frequency list associated with the list

\sphinxlineitem{huffman\_table: dict}
\sphinxAtStartPar
initialize as an empty dictionary. Datastructure for huffman code

\end{description}

\end{description}
\index{decode() (huffman.Huffman static method)@\spxentry{decode()}\spxextra{huffman.Huffman static method}}

\begin{fulllineitems}
\phantomsection\label{\detokenize{huffman:huffman.Huffman.decode}}
\pysigstartsignatures
\pysiglinewithargsret{\sphinxbfcode{\sphinxupquote{static\DUrole{w}{  }}}\sphinxbfcode{\sphinxupquote{decode}}}{\emph{\DUrole{n}{encoded\_value}\DUrole{p}{:}\DUrole{w}{  }\DUrole{n}{str}}, \emph{\DUrole{n}{root\_node}\DUrole{p}{:}\DUrole{w}{  }\DUrole{n}{{\hyperref[\detokenize{huffman:huffman.Node}]{\sphinxcrossref{Node}}}}}}{{ $\rightarrow$ str}}
\pysigstopsignatures
\sphinxAtStartPar
Decodes the encoded value into a set of symbols
\begin{description}
\sphinxlineitem{Parameters:}\begin{description}
\sphinxlineitem{encoded\_value: str}
\sphinxAtStartPar
encoded value is a string of binary

\sphinxlineitem{root\_node: Node}
\sphinxAtStartPar
root node of the huffman tree. Traverses through the node to decode.

\end{description}

\sphinxlineitem{Returns:}\begin{description}
\sphinxlineitem{decoded\_symbols: str}
\sphinxAtStartPar
the symbols after decoding using the huffman tree.

\end{description}

\end{description}

\begin{sphinxVerbatim}[commandchars=\\\{\}]
\PYG{g+gp}{\PYGZgt{}\PYGZgt{}\PYGZgt{} }\PYG{n}{h} \PYG{o}{=} \PYG{n}{Huffman}\PYG{p}{(}\PYG{p}{[}\PYG{l+s+s1}{\PYGZsq{}}\PYG{l+s+s1}{a}\PYG{l+s+s1}{\PYGZsq{}}\PYG{p}{,} \PYG{l+s+s1}{\PYGZsq{}}\PYG{l+s+s1}{b}\PYG{l+s+s1}{\PYGZsq{}}\PYG{p}{,} \PYG{l+s+s1}{\PYGZsq{}}\PYG{l+s+s1}{c}\PYG{l+s+s1}{\PYGZsq{}}\PYG{p}{,} \PYG{l+s+s1}{\PYGZsq{}}\PYG{l+s+s1}{d}\PYG{l+s+s1}{\PYGZsq{}}\PYG{p}{,} \PYG{l+s+s1}{\PYGZsq{}}\PYG{l+s+s1}{e}\PYG{l+s+s1}{\PYGZsq{}}\PYG{p}{]}\PYG{p}{,} \PYG{p}{[}\PYG{l+m+mf}{0.3}\PYG{p}{,} \PYG{l+m+mf}{0.25}\PYG{p}{,} \PYG{l+m+mf}{0.2}\PYG{p}{,} \PYG{l+m+mf}{0.15}\PYG{p}{,} \PYG{l+m+mf}{0.1}\PYG{p}{]}\PYG{p}{)}
\PYG{g+gp}{\PYGZgt{}\PYGZgt{}\PYGZgt{} }\PYG{n}{enc\PYGZus{}value}\PYG{p}{,} \PYG{n}{root\PYGZus{}node} \PYG{o}{=} \PYG{n}{h}\PYG{o}{.}\PYG{n}{encode}\PYG{p}{(}\PYG{p}{[}\PYG{l+s+s1}{\PYGZsq{}}\PYG{l+s+s1}{a}\PYG{l+s+s1}{\PYGZsq{}}\PYG{p}{,} \PYG{l+s+s1}{\PYGZsq{}}\PYG{l+s+s1}{b}\PYG{l+s+s1}{\PYGZsq{}}\PYG{p}{,} \PYG{l+s+s1}{\PYGZsq{}}\PYG{l+s+s1}{c}\PYG{l+s+s1}{\PYGZsq{}}\PYG{p}{,} \PYG{l+s+s1}{\PYGZsq{}}\PYG{l+s+s1}{b}\PYG{l+s+s1}{\PYGZsq{}}\PYG{p}{,} \PYG{l+s+s1}{\PYGZsq{}}\PYG{l+s+s1}{c}\PYG{l+s+s1}{\PYGZsq{}}\PYG{p}{,} \PYG{l+s+s1}{\PYGZsq{}}\PYG{l+s+s1}{e}\PYG{l+s+s1}{\PYGZsq{}}\PYG{p}{]}\PYG{p}{)}
\PYG{g+gp}{\PYGZgt{}\PYGZgt{}\PYGZgt{} }\PYG{n+nb}{print}\PYG{p}{(}\PYG{n}{h}\PYG{o}{.}\PYG{n}{decode}\PYG{p}{(}\PYG{n}{enc\PYGZus{}value}\PYG{p}{,} \PYG{n}{root\PYGZus{}node}\PYG{p}{)}\PYG{p}{)}
\PYG{g+go}{abcbce}
\end{sphinxVerbatim}

\end{fulllineitems}

\index{encode() (huffman.Huffman method)@\spxentry{encode()}\spxextra{huffman.Huffman method}}

\begin{fulllineitems}
\phantomsection\label{\detokenize{huffman:huffman.Huffman.encode}}
\pysigstartsignatures
\pysiglinewithargsret{\sphinxbfcode{\sphinxupquote{encode}}}{\emph{\DUrole{n}{msg}\DUrole{p}{:}\DUrole{w}{  }\DUrole{n}{list}}}{{ $\rightarrow$ Tuple\DUrole{p}{{[}}str\DUrole{p}{,}\DUrole{w}{  }{\hyperref[\detokenize{huffman:huffman.Node}]{\sphinxcrossref{Node}}}\DUrole{p}{{]}}}}
\pysigstopsignatures
\sphinxAtStartPar
Using the huffman’s table encodes a message
\begin{description}
\sphinxlineitem{Parameters:}\begin{description}
\sphinxlineitem{msg: list}
\sphinxAtStartPar
a list of symbols

\end{description}

\sphinxlineitem{Returns:}\begin{description}
\sphinxlineitem{encoded\_value: str}
\sphinxAtStartPar
binary string after encoding the message

\sphinxlineitem{root\_node: Node}
\sphinxAtStartPar
root\_node of the huffman tree.

\end{description}

\sphinxlineitem{Raises: }
\sphinxAtStartPar
Assertion error: If message contains invalid symbol

\end{description}

\begin{sphinxVerbatim}[commandchars=\\\{\}]
\PYG{g+gp}{\PYGZgt{}\PYGZgt{}\PYGZgt{} }\PYG{n}{h} \PYG{o}{=} \PYG{n}{Huffman}\PYG{p}{(}\PYG{p}{[}\PYG{l+s+s1}{\PYGZsq{}}\PYG{l+s+s1}{a}\PYG{l+s+s1}{\PYGZsq{}}\PYG{p}{,} \PYG{l+s+s1}{\PYGZsq{}}\PYG{l+s+s1}{b}\PYG{l+s+s1}{\PYGZsq{}}\PYG{p}{,} \PYG{l+s+s1}{\PYGZsq{}}\PYG{l+s+s1}{c}\PYG{l+s+s1}{\PYGZsq{}}\PYG{p}{,} \PYG{l+s+s1}{\PYGZsq{}}\PYG{l+s+s1}{d}\PYG{l+s+s1}{\PYGZsq{}}\PYG{p}{,} \PYG{l+s+s1}{\PYGZsq{}}\PYG{l+s+s1}{e}\PYG{l+s+s1}{\PYGZsq{}}\PYG{p}{]}\PYG{p}{,} \PYG{p}{[}\PYG{l+m+mf}{0.3}\PYG{p}{,} \PYG{l+m+mf}{0.25}\PYG{p}{,} \PYG{l+m+mf}{0.2}\PYG{p}{,} \PYG{l+m+mf}{0.15}\PYG{p}{,} \PYG{l+m+mf}{0.1}\PYG{p}{]}\PYG{p}{)}
\PYG{g+gp}{\PYGZgt{}\PYGZgt{}\PYGZgt{} }\PYG{n}{enc\PYGZus{}value}\PYG{p}{,} \PYG{n}{root\PYGZus{}node} \PYG{o}{=} \PYG{n}{h}\PYG{o}{.}\PYG{n}{encode}\PYG{p}{(}\PYG{p}{[}\PYG{l+s+s1}{\PYGZsq{}}\PYG{l+s+s1}{a}\PYG{l+s+s1}{\PYGZsq{}}\PYG{p}{,} \PYG{l+s+s1}{\PYGZsq{}}\PYG{l+s+s1}{b}\PYG{l+s+s1}{\PYGZsq{}}\PYG{p}{,} \PYG{l+s+s1}{\PYGZsq{}}\PYG{l+s+s1}{c}\PYG{l+s+s1}{\PYGZsq{}}\PYG{p}{,} \PYG{l+s+s1}{\PYGZsq{}}\PYG{l+s+s1}{b}\PYG{l+s+s1}{\PYGZsq{}}\PYG{p}{,} \PYG{l+s+s1}{\PYGZsq{}}\PYG{l+s+s1}{c}\PYG{l+s+s1}{\PYGZsq{}}\PYG{p}{,} \PYG{l+s+s1}{\PYGZsq{}}\PYG{l+s+s1}{e}\PYG{l+s+s1}{\PYGZsq{}}\PYG{p}{]}\PYG{p}{)}
\PYG{g+gp}{\PYGZgt{}\PYGZgt{}\PYGZgt{} }\PYG{n+nb}{print}\PYG{p}{(}\PYG{n}{enc\PYGZus{}value}\PYG{p}{)}
\PYG{g+go}{b0010111011011}
\end{sphinxVerbatim}

\end{fulllineitems}

\index{huffman\_tree() (huffman.Huffman method)@\spxentry{huffman\_tree()}\spxextra{huffman.Huffman method}}

\begin{fulllineitems}
\phantomsection\label{\detokenize{huffman:huffman.Huffman.huffman_tree}}
\pysigstartsignatures
\pysiglinewithargsret{\sphinxbfcode{\sphinxupquote{huffman\_tree}}}{}{{ $\rightarrow$ {\hyperref[\detokenize{huffman:huffman.Node}]{\sphinxcrossref{Node}}}}}
\pysigstopsignatures
\sphinxAtStartPar
Using the recursive huffman’s technique creates a huffman tree.
\begin{description}
\sphinxlineitem{Returns:}\begin{description}
\sphinxlineitem{root\_node: Node}
\sphinxAtStartPar
returns the root node. The tree can be constructed from the root node as it internal nodes are childrens.

\end{description}

\end{description}

\end{fulllineitems}

\index{print\_tree() (huffman.Huffman static method)@\spxentry{print\_tree()}\spxextra{huffman.Huffman static method}}

\begin{fulllineitems}
\phantomsection\label{\detokenize{huffman:huffman.Huffman.print_tree}}
\pysigstartsignatures
\pysiglinewithargsret{\sphinxbfcode{\sphinxupquote{static\DUrole{w}{  }}}\sphinxbfcode{\sphinxupquote{print\_tree}}}{\emph{\DUrole{n}{root}\DUrole{p}{:}\DUrole{w}{  }\DUrole{n}{{\hyperref[\detokenize{huffman:huffman.Node}]{\sphinxcrossref{Node}}}}}}{{ $\rightarrow$ None}}
\pysigstopsignatures
\sphinxAtStartPar
Prints a tree in the terminal given a node
\begin{description}
\sphinxlineitem{Parameters:}\begin{description}
\sphinxlineitem{root: Node}
\sphinxAtStartPar
pass in a object of class Node prints a tree.

\end{description}

\end{description}

\begin{sphinxVerbatim}[commandchars=\\\{\}]
\PYG{g+gp}{\PYGZgt{}\PYGZgt{}\PYGZgt{} }\PYG{n}{h} \PYG{o}{=} \PYG{n}{Huffman}\PYG{p}{(}\PYG{p}{[}\PYG{l+s+s1}{\PYGZsq{}}\PYG{l+s+s1}{a}\PYG{l+s+s1}{\PYGZsq{}}\PYG{p}{,} \PYG{l+s+s1}{\PYGZsq{}}\PYG{l+s+s1}{b}\PYG{l+s+s1}{\PYGZsq{}}\PYG{p}{,} \PYG{l+s+s1}{\PYGZsq{}}\PYG{l+s+s1}{c}\PYG{l+s+s1}{\PYGZsq{}}\PYG{p}{,} \PYG{l+s+s1}{\PYGZsq{}}\PYG{l+s+s1}{d}\PYG{l+s+s1}{\PYGZsq{}}\PYG{p}{,} \PYG{l+s+s1}{\PYGZsq{}}\PYG{l+s+s1}{e}\PYG{l+s+s1}{\PYGZsq{}}\PYG{p}{]}\PYG{p}{,} \PYG{p}{[}\PYG{l+m+mf}{0.3}\PYG{p}{,} \PYG{l+m+mf}{0.25}\PYG{p}{,} \PYG{l+m+mf}{0.2}\PYG{p}{,} \PYG{l+m+mf}{0.15}\PYG{p}{,} \PYG{l+m+mf}{0.1}\PYG{p}{]}\PYG{p}{)}
\PYG{g+gp}{\PYGZgt{}\PYGZgt{}\PYGZgt{} }\PYG{n}{\PYGZus{}}\PYG{p}{,} \PYG{n}{root\PYGZus{}node} \PYG{o}{=} \PYG{n}{h}\PYG{o}{.}\PYG{n}{encode}\PYG{p}{(}\PYG{p}{[}\PYG{l+s+s1}{\PYGZsq{}}\PYG{l+s+s1}{a}\PYG{l+s+s1}{\PYGZsq{}}\PYG{p}{,} \PYG{l+s+s1}{\PYGZsq{}}\PYG{l+s+s1}{b}\PYG{l+s+s1}{\PYGZsq{}}\PYG{p}{,} \PYG{l+s+s1}{\PYGZsq{}}\PYG{l+s+s1}{c}\PYG{l+s+s1}{\PYGZsq{}}\PYG{p}{,} \PYG{l+s+s1}{\PYGZsq{}}\PYG{l+s+s1}{b}\PYG{l+s+s1}{\PYGZsq{}}\PYG{p}{,} \PYG{l+s+s1}{\PYGZsq{}}\PYG{l+s+s1}{c}\PYG{l+s+s1}{\PYGZsq{}}\PYG{p}{,} \PYG{l+s+s1}{\PYGZsq{}}\PYG{l+s+s1}{e}\PYG{l+s+s1}{\PYGZsq{}}\PYG{p}{]}\PYG{p}{)}
\PYG{g+gp}{\PYGZgt{}\PYGZgt{}\PYGZgt{} }\PYG{n}{h}\PYG{o}{.}\PYG{n}{print\PYGZus{}tree}\PYG{p}{(}\PYG{n}{root\PYGZus{}node}\PYG{p}{)}
\PYG{g+go}{((a(de))(bc))}
\PYG{g+go}{    /¯¯¯¯¯¯   ¯¯¯¯¯¯\PYGZbs{}}
\PYG{g+go}{(a(de))            (bc)}
\PYG{g+go}{/¯¯¯ ¯¯¯\PYGZbs{}       /¯¯¯ ¯¯¯\PYGZbs{}}
\PYG{g+go}{a    (de)       b       c}
\PYG{g+go}{        /¯ ¯\PYGZbs{}}
\PYG{g+go}{        d   e}
\end{sphinxVerbatim}

\end{fulllineitems}

\index{show\_table() (huffman.Huffman method)@\spxentry{show\_table()}\spxextra{huffman.Huffman method}}

\begin{fulllineitems}
\phantomsection\label{\detokenize{huffman:huffman.Huffman.show_table}}
\pysigstartsignatures
\pysiglinewithargsret{\sphinxbfcode{\sphinxupquote{show\_table}}}{}{{ $\rightarrow$ None}}
\pysigstopsignatures
\sphinxAtStartPar
Prints the huffman encoding table.
calls the huffman tree and get\_codes function.

\begin{sphinxVerbatim}[commandchars=\\\{\}]
\PYG{g+gp}{\PYGZgt{}\PYGZgt{}\PYGZgt{} }\PYG{n}{h} \PYG{o}{=} \PYG{n}{Huffman}\PYG{p}{(}\PYG{p}{[}\PYG{l+s+s1}{\PYGZsq{}}\PYG{l+s+s1}{a}\PYG{l+s+s1}{\PYGZsq{}}\PYG{p}{,} \PYG{l+s+s1}{\PYGZsq{}}\PYG{l+s+s1}{b}\PYG{l+s+s1}{\PYGZsq{}}\PYG{p}{,} \PYG{l+s+s1}{\PYGZsq{}}\PYG{l+s+s1}{c}\PYG{l+s+s1}{\PYGZsq{}}\PYG{p}{,} \PYG{l+s+s1}{\PYGZsq{}}\PYG{l+s+s1}{d}\PYG{l+s+s1}{\PYGZsq{}}\PYG{p}{,} \PYG{l+s+s1}{\PYGZsq{}}\PYG{l+s+s1}{e}\PYG{l+s+s1}{\PYGZsq{}}\PYG{p}{]}\PYG{p}{,} \PYG{p}{[}\PYG{l+m+mf}{0.3}\PYG{p}{,} \PYG{l+m+mf}{0.25}\PYG{p}{,} \PYG{l+m+mf}{0.2}\PYG{p}{,} \PYG{l+m+mf}{0.15}\PYG{p}{,} \PYG{l+m+mf}{0.1}\PYG{p}{]}\PYG{p}{)}
\PYG{g+gp}{\PYGZgt{}\PYGZgt{}\PYGZgt{} }\PYG{n}{h}\PYG{o}{.}\PYG{n}{show\PYGZus{}table}\PYG{p}{(}\PYG{p}{)}
\PYG{g+go}{+\PYGZhy{}\PYGZhy{}\PYGZhy{}\PYGZhy{}\PYGZhy{}\PYGZhy{}\PYGZhy{}\PYGZhy{}\PYGZhy{}+\PYGZhy{}\PYGZhy{}\PYGZhy{}\PYGZhy{}\PYGZhy{}\PYGZhy{}\PYGZhy{}\PYGZhy{}\PYGZhy{}\PYGZhy{}\PYGZhy{}+}
\PYG{g+go}{| Symbols | Codewords |}
\PYG{g+go}{+\PYGZhy{}\PYGZhy{}\PYGZhy{}\PYGZhy{}\PYGZhy{}\PYGZhy{}\PYGZhy{}\PYGZhy{}\PYGZhy{}+\PYGZhy{}\PYGZhy{}\PYGZhy{}\PYGZhy{}\PYGZhy{}\PYGZhy{}\PYGZhy{}\PYGZhy{}\PYGZhy{}\PYGZhy{}\PYGZhy{}+}
\PYG{g+go}{|    a    |    00     |}
\PYG{g+go}{|    d    |    010    |}
\PYG{g+go}{|    e    |    011    |}
\PYG{g+go}{|    b    |    10     |}
\PYG{g+go}{|    c    |    11     |}
\PYG{g+go}{+\PYGZhy{}\PYGZhy{}\PYGZhy{}\PYGZhy{}\PYGZhy{}\PYGZhy{}\PYGZhy{}\PYGZhy{}\PYGZhy{}+\PYGZhy{}\PYGZhy{}\PYGZhy{}\PYGZhy{}\PYGZhy{}\PYGZhy{}\PYGZhy{}\PYGZhy{}\PYGZhy{}\PYGZhy{}\PYGZhy{}+        }
\end{sphinxVerbatim}

\end{fulllineitems}


\end{fulllineitems}

\index{Node (class in huffman)@\spxentry{Node}\spxextra{class in huffman}}

\begin{fulllineitems}
\phantomsection\label{\detokenize{huffman:huffman.Node}}
\pysigstartsignatures
\pysiglinewithargsret{\sphinxbfcode{\sphinxupquote{class\DUrole{w}{  }}}\sphinxcode{\sphinxupquote{huffman.}}\sphinxbfcode{\sphinxupquote{Node}}}{\emph{\DUrole{n}{symbol}\DUrole{p}{:}\DUrole{w}{  }\DUrole{n}{str}}, \emph{\DUrole{n}{freq}\DUrole{p}{:}\DUrole{w}{  }\DUrole{n}{Tuple\DUrole{p}{{[}}int\DUrole{p}{,}\DUrole{w}{  }float\DUrole{p}{,}\DUrole{w}{  }numpy.int32\DUrole{p}{{]}}}}, \emph{\DUrole{n}{left}\DUrole{o}{=}\DUrole{default_value}{None}}, \emph{\DUrole{n}{right}\DUrole{o}{=}\DUrole{default_value}{None}}}{}
\pysigstopsignatures
\sphinxAtStartPar
Bases: \sphinxcode{\sphinxupquote{object}}

\sphinxAtStartPar
Class for a node.
\begin{description}
\sphinxlineitem{Attributes:}\begin{description}
\sphinxlineitem{symbols: str}
\sphinxAtStartPar
symbol of the node

\sphinxlineitem{freq: int | float}
\sphinxAtStartPar
frequency of the node. For internal nodes calculated as sum of nodes

\sphinxlineitem{left: Node, default: None}
\sphinxAtStartPar
left child of a node

\sphinxlineitem{right: Node, default: None}
\sphinxAtStartPar
right child of a node

\end{description}

\end{description}

\end{fulllineitems}


\sphinxstepscope


\section{Arithmetic Coding}
\label{\detokenize{arithmetic_coding:module-arithmetic_coding}}\label{\detokenize{arithmetic_coding:arithmetic-coding}}\label{\detokenize{arithmetic_coding::doc}}\index{module@\spxentry{module}!arithmetic\_coding@\spxentry{arithmetic\_coding}}\index{arithmetic\_coding@\spxentry{arithmetic\_coding}!module@\spxentry{module}}\index{ArithmeticCoding (class in arithmetic\_coding)@\spxentry{ArithmeticCoding}\spxextra{class in arithmetic\_coding}}

\begin{fulllineitems}
\phantomsection\label{\detokenize{arithmetic_coding:arithmetic_coding.ArithmeticCoding}}
\pysigstartsignatures
\pysiglinewithargsret{\sphinxbfcode{\sphinxupquote{class\DUrole{w}{  }}}\sphinxcode{\sphinxupquote{arithmetic\_coding.}}\sphinxbfcode{\sphinxupquote{ArithmeticCoding}}}{\emph{\DUrole{n}{symbols}\DUrole{p}{:}\DUrole{w}{  }\DUrole{n}{list}}, \emph{\DUrole{n}{frequency}\DUrole{p}{:}\DUrole{w}{  }\DUrole{n}{list}}, \emph{\DUrole{n}{message}\DUrole{p}{:}\DUrole{w}{  }\DUrole{n}{list}\DUrole{w}{  }\DUrole{o}{=}\DUrole{w}{  }\DUrole{default_value}{None}}}{}
\pysigstopsignatures
\sphinxAtStartPar
Bases: {\hyperref[\detokenize{core:core.data.Data}]{\sphinxcrossref{\sphinxcode{\sphinxupquote{Data}}}}}

\sphinxAtStartPar
Class for arithmetic coding compression with out rescaling. Might not be efficient. Inherits the data class. Allows compression in two ways.
\begin{enumerate}
\sphinxsetlistlabels{\arabic}{enumi}{enumii}{}{.}%
\item {} \begin{description}
\sphinxlineitem{By specifying the symbols and frequency.}
\sphinxAtStartPar
In this case arg:msg must be provided in the encode step.

\end{description}

\item {} \begin{description}
\sphinxlineitem{By giving the entire message itself. }
\sphinxAtStartPar
No argument required in the encode step. 
Computes the probability and cumulative distribution from the message itself.

\end{description}

\end{enumerate}
\begin{description}
\sphinxlineitem{Attributes:}\begin{description}
\sphinxlineitem{symbols}{[}list{]}
\sphinxAtStartPar
list of symbols, elements can be any format

\sphinxlineitem{frequency: list}
\sphinxAtStartPar
frequency list associated with the list

\sphinxlineitem{message: list, default = None}
\sphinxAtStartPar
list of message.

\end{description}

\end{description}
\index{decode() (arithmetic\_coding.ArithmeticCoding method)@\spxentry{decode()}\spxextra{arithmetic\_coding.ArithmeticCoding method}}

\begin{fulllineitems}
\phantomsection\label{\detokenize{arithmetic_coding:arithmetic_coding.ArithmeticCoding.decode}}
\pysigstartsignatures
\pysiglinewithargsret{\sphinxbfcode{\sphinxupquote{decode}}}{\emph{\DUrole{n}{encoded\_value}\DUrole{p}{:}\DUrole{w}{  }\DUrole{n}{bitarray.bitarray}}, \emph{\DUrole{n}{msg\_length}\DUrole{p}{:}\DUrole{w}{  }\DUrole{n}{int}}, \emph{\DUrole{n}{show\_steps}\DUrole{p}{:}\DUrole{w}{  }\DUrole{n}{bool}\DUrole{w}{  }\DUrole{o}{=}\DUrole{w}{  }\DUrole{default_value}{False}}}{}
\pysigstopsignatures
\sphinxAtStartPar
Using the decoding by checking interval, updating interval, and picking new symbol.
\begin{description}
\sphinxlineitem{Parameters: }\begin{description}
\sphinxlineitem{encoded\_value: bitarray.bitarray}
\sphinxAtStartPar
bitarray instance of the encoded value.

\sphinxlineitem{msg\_length: int}
\sphinxAtStartPar
length of the message. needs to be specified to the same number as original msg to get right decoding.

\sphinxlineitem{show\_steps: bool, default = False}
\sphinxAtStartPar
shows decoding steps.

\end{description}

\sphinxlineitem{Returns:}\begin{description}
\sphinxlineitem{decoded\_symbols: str}
\sphinxAtStartPar
returns the decoded symbols

\end{description}

\end{description}

\begin{sphinxVerbatim}[commandchars=\\\{\}]
\PYG{g+gp}{\PYGZgt{}\PYGZgt{}\PYGZgt{} }\PYG{n}{symbols} \PYG{o}{=} \PYG{p}{[}\PYG{l+s+s1}{\PYGZsq{}}\PYG{l+s+s1}{a}\PYG{l+s+s1}{\PYGZsq{}}\PYG{p}{,} \PYG{l+s+s1}{\PYGZsq{}}\PYG{l+s+s1}{b}\PYG{l+s+s1}{\PYGZsq{}}\PYG{p}{,} \PYG{l+s+s1}{\PYGZsq{}}\PYG{l+s+s1}{c}\PYG{l+s+s1}{\PYGZsq{}}\PYG{p}{]}
\PYG{g+gp}{\PYGZgt{}\PYGZgt{}\PYGZgt{} }\PYG{n}{freq} \PYG{o}{=} \PYG{p}{[}\PYG{l+m+mf}{0.8}\PYG{p}{,} \PYG{l+m+mf}{0.02}\PYG{p}{,} \PYG{l+m+mf}{0.18}\PYG{p}{]}
\PYG{g+gp}{\PYGZgt{}\PYGZgt{}\PYGZgt{} }\PYG{n}{f} \PYG{o}{=} \PYG{n}{ArithmeticCoding}\PYG{p}{(}\PYG{n}{symbols}\PYG{p}{,} \PYG{n}{freq}\PYG{p}{)}
\PYG{g+gp}{\PYGZgt{}\PYGZgt{}\PYGZgt{} }\PYG{n}{encoded\PYGZus{}value}\PYG{p}{,} \PYG{n}{msg\PYGZus{}len} \PYG{o}{=} \PYG{n}{f}\PYG{o}{.}\PYG{n}{encode}\PYG{p}{(}\PYG{l+s+s1}{\PYGZsq{}}\PYG{l+s+s1}{abaca}\PYG{l+s+s1}{\PYGZsq{}}\PYG{p}{)}
\PYG{g+gp}{\PYGZgt{}\PYGZgt{}\PYGZgt{} }\PYG{n}{decoded\PYGZus{}value} \PYG{o}{=} \PYG{n}{f}\PYG{o}{.}\PYG{n}{decode}\PYG{p}{(}\PYG{n}{encoded\PYGZus{}value}\PYG{p}{,} \PYG{n}{msg\PYGZus{}len}\PYG{p}{)}
\PYG{g+gp}{\PYGZgt{}\PYGZgt{}\PYGZgt{} }\PYG{n+nb}{print}\PYG{p}{(}\PYG{n}{decoded\PYGZus{}value}\PYG{p}{)}
\PYG{g+go}{abaca}
\end{sphinxVerbatim}

\begin{sphinxVerbatim}[commandchars=\\\{\}]
\PYG{g+gp}{\PYGZgt{}\PYGZgt{}\PYGZgt{} }\PYG{n}{decoded\PYGZus{}value} \PYG{o}{=} \PYG{n}{f}\PYG{o}{.}\PYG{n}{decode}\PYG{p}{(}\PYG{n}{encoded\PYGZus{}value}\PYG{p}{,} \PYG{n}{msg\PYGZus{}len}\PYG{p}{,} \PYG{n}{show\PYGZus{}steps}\PYG{o}{=}\PYG{k+kc}{True}\PYG{p}{)}
\PYG{g+go}{Decoding Process}
\PYG{g+go}{\PYGZhy{}\PYGZhy{}\PYGZhy{}\PYGZhy{}\PYGZhy{}\PYGZhy{}\PYGZhy{}\PYGZhy{}\PYGZhy{}\PYGZhy{}\PYGZhy{}\PYGZhy{}\PYGZhy{}\PYGZhy{}\PYGZhy{}\PYGZhy{}\PYGZhy{}\PYGZhy{}}
\PYG{g+go}{+\PYGZhy{}\PYGZhy{}\PYGZhy{}\PYGZhy{}\PYGZhy{}\PYGZhy{}\PYGZhy{}\PYGZhy{}\PYGZhy{}\PYGZhy{}\PYGZhy{}\PYGZhy{}\PYGZhy{}\PYGZhy{}+\PYGZhy{}\PYGZhy{}\PYGZhy{}\PYGZhy{}\PYGZhy{}\PYGZhy{}\PYGZhy{}\PYGZhy{}\PYGZhy{}\PYGZhy{}\PYGZhy{}\PYGZhy{}\PYGZhy{}\PYGZhy{}\PYGZhy{}\PYGZhy{}\PYGZhy{}\PYGZhy{}\PYGZhy{}\PYGZhy{}\PYGZhy{}\PYGZhy{}\PYGZhy{}\PYGZhy{}\PYGZhy{}+\PYGZhy{}\PYGZhy{}\PYGZhy{}\PYGZhy{}\PYGZhy{}\PYGZhy{}\PYGZhy{}\PYGZhy{}\PYGZhy{}\PYGZhy{}\PYGZhy{}\PYGZhy{}\PYGZhy{}\PYGZhy{}\PYGZhy{}+\PYGZhy{}\PYGZhy{}\PYGZhy{}\PYGZhy{}\PYGZhy{}\PYGZhy{}\PYGZhy{}\PYGZhy{}\PYGZhy{}\PYGZhy{}\PYGZhy{}\PYGZhy{}\PYGZhy{}\PYGZhy{}\PYGZhy{}\PYGZhy{}\PYGZhy{}\PYGZhy{}\PYGZhy{}\PYGZhy{}\PYGZhy{}\PYGZhy{}\PYGZhy{}\PYGZhy{}\PYGZhy{}\PYGZhy{}\PYGZhy{}\PYGZhy{}\PYGZhy{}\PYGZhy{}\PYGZhy{}\PYGZhy{}\PYGZhy{}\PYGZhy{}\PYGZhy{}\PYGZhy{}\PYGZhy{}\PYGZhy{}\PYGZhy{}\PYGZhy{}\PYGZhy{}\PYGZhy{}\PYGZhy{}\PYGZhy{}+\PYGZhy{}\PYGZhy{}\PYGZhy{}\PYGZhy{}\PYGZhy{}\PYGZhy{}\PYGZhy{}\PYGZhy{}\PYGZhy{}\PYGZhy{}\PYGZhy{}+}
\PYG{g+go}{| Decoded Symb |      Encoded Value      |      Tag      |                   Range                    |  Remark   |}
\PYG{g+go}{+\PYGZhy{}\PYGZhy{}\PYGZhy{}\PYGZhy{}\PYGZhy{}\PYGZhy{}\PYGZhy{}\PYGZhy{}\PYGZhy{}\PYGZhy{}\PYGZhy{}\PYGZhy{}\PYGZhy{}\PYGZhy{}+\PYGZhy{}\PYGZhy{}\PYGZhy{}\PYGZhy{}\PYGZhy{}\PYGZhy{}\PYGZhy{}\PYGZhy{}\PYGZhy{}\PYGZhy{}\PYGZhy{}\PYGZhy{}\PYGZhy{}\PYGZhy{}\PYGZhy{}\PYGZhy{}\PYGZhy{}\PYGZhy{}\PYGZhy{}\PYGZhy{}\PYGZhy{}\PYGZhy{}\PYGZhy{}\PYGZhy{}\PYGZhy{}+\PYGZhy{}\PYGZhy{}\PYGZhy{}\PYGZhy{}\PYGZhy{}\PYGZhy{}\PYGZhy{}\PYGZhy{}\PYGZhy{}\PYGZhy{}\PYGZhy{}\PYGZhy{}\PYGZhy{}\PYGZhy{}\PYGZhy{}+\PYGZhy{}\PYGZhy{}\PYGZhy{}\PYGZhy{}\PYGZhy{}\PYGZhy{}\PYGZhy{}\PYGZhy{}\PYGZhy{}\PYGZhy{}\PYGZhy{}\PYGZhy{}\PYGZhy{}\PYGZhy{}\PYGZhy{}\PYGZhy{}\PYGZhy{}\PYGZhy{}\PYGZhy{}\PYGZhy{}\PYGZhy{}\PYGZhy{}\PYGZhy{}\PYGZhy{}\PYGZhy{}\PYGZhy{}\PYGZhy{}\PYGZhy{}\PYGZhy{}\PYGZhy{}\PYGZhy{}\PYGZhy{}\PYGZhy{}\PYGZhy{}\PYGZhy{}\PYGZhy{}\PYGZhy{}\PYGZhy{}\PYGZhy{}\PYGZhy{}\PYGZhy{}\PYGZhy{}\PYGZhy{}\PYGZhy{}+\PYGZhy{}\PYGZhy{}\PYGZhy{}\PYGZhy{}\PYGZhy{}\PYGZhy{}\PYGZhy{}\PYGZhy{}\PYGZhy{}\PYGZhy{}\PYGZhy{}+}
\PYG{g+go}{|      h       | bitarray(\PYGZsq{}00010001011\PYGZsq{}) | 0.06787109375 |                  (0, 0.2)                  | Pick next |}
\PYG{g+go}{|      he      | bitarray(\PYGZsq{}00010001011\PYGZsq{}) | 0.06787109375 | (0.04000000000000001, 0.08000000000000002) | Pick next |}
\PYG{g+go}{|     hel      | bitarray(\PYGZsq{}00010001011\PYGZsq{}) | 0.06787109375 | (0.05600000000000001, 0.07200000000000001) | Pick next |}
\PYG{g+go}{|     hell     | bitarray(\PYGZsq{}00010001011\PYGZsq{}) | 0.06787109375 | (0.06240000000000001, 0.06880000000000001) | Pick next |}
\PYG{g+go}{|    hello     | bitarray(\PYGZsq{}00010001011\PYGZsq{}) | 0.06787109375 | (0.06752000000000001, 0.06880000000000001) | Pick next |}
\PYG{g+go}{+\PYGZhy{}\PYGZhy{}\PYGZhy{}\PYGZhy{}\PYGZhy{}\PYGZhy{}\PYGZhy{}\PYGZhy{}\PYGZhy{}\PYGZhy{}\PYGZhy{}\PYGZhy{}\PYGZhy{}\PYGZhy{}+\PYGZhy{}\PYGZhy{}\PYGZhy{}\PYGZhy{}\PYGZhy{}\PYGZhy{}\PYGZhy{}\PYGZhy{}\PYGZhy{}\PYGZhy{}\PYGZhy{}\PYGZhy{}\PYGZhy{}\PYGZhy{}\PYGZhy{}\PYGZhy{}\PYGZhy{}\PYGZhy{}\PYGZhy{}\PYGZhy{}\PYGZhy{}\PYGZhy{}\PYGZhy{}\PYGZhy{}\PYGZhy{}+\PYGZhy{}\PYGZhy{}\PYGZhy{}\PYGZhy{}\PYGZhy{}\PYGZhy{}\PYGZhy{}\PYGZhy{}\PYGZhy{}\PYGZhy{}\PYGZhy{}\PYGZhy{}\PYGZhy{}\PYGZhy{}\PYGZhy{}+\PYGZhy{}\PYGZhy{}\PYGZhy{}\PYGZhy{}\PYGZhy{}\PYGZhy{}\PYGZhy{}\PYGZhy{}\PYGZhy{}\PYGZhy{}\PYGZhy{}\PYGZhy{}\PYGZhy{}\PYGZhy{}\PYGZhy{}\PYGZhy{}\PYGZhy{}\PYGZhy{}\PYGZhy{}\PYGZhy{}\PYGZhy{}\PYGZhy{}\PYGZhy{}\PYGZhy{}\PYGZhy{}\PYGZhy{}\PYGZhy{}\PYGZhy{}\PYGZhy{}\PYGZhy{}\PYGZhy{}\PYGZhy{}\PYGZhy{}\PYGZhy{}\PYGZhy{}\PYGZhy{}\PYGZhy{}\PYGZhy{}\PYGZhy{}\PYGZhy{}\PYGZhy{}\PYGZhy{}\PYGZhy{}\PYGZhy{}+\PYGZhy{}\PYGZhy{}\PYGZhy{}\PYGZhy{}\PYGZhy{}\PYGZhy{}\PYGZhy{}\PYGZhy{}\PYGZhy{}\PYGZhy{}\PYGZhy{}+}
\PYG{g+go}{Decoded Value = hello}
\end{sphinxVerbatim}

\end{fulllineitems}

\index{encode() (arithmetic\_coding.ArithmeticCoding method)@\spxentry{encode()}\spxextra{arithmetic\_coding.ArithmeticCoding method}}

\begin{fulllineitems}
\phantomsection\label{\detokenize{arithmetic_coding:arithmetic_coding.ArithmeticCoding.encode}}
\pysigstartsignatures
\pysiglinewithargsret{\sphinxbfcode{\sphinxupquote{encode}}}{\emph{\DUrole{n}{msg}\DUrole{p}{:}\DUrole{w}{  }\DUrole{n}{list}\DUrole{w}{  }\DUrole{o}{=}\DUrole{w}{  }\DUrole{default_value}{None}}, \emph{\DUrole{n}{show\_steps}\DUrole{p}{:}\DUrole{w}{  }\DUrole{n}{bool}\DUrole{w}{  }\DUrole{o}{=}\DUrole{w}{  }\DUrole{default_value}{False}}}{{ $\rightarrow$ Tuple\DUrole{p}{{[}}bitarray.bitarray\DUrole{p}{,}\DUrole{w}{  }int\DUrole{p}{{]}}}}
\pysigstopsignatures
\sphinxAtStartPar
Arithmetic encoding function. Can be used to encdoe in either ways as specifiec before.
\begin{description}
\sphinxlineitem{Parameters:}\begin{description}
\sphinxlineitem{msg: list, default = None}
\sphinxAtStartPar
message you want to encode

\sphinxlineitem{show\_steps: bool, default = False}
\sphinxAtStartPar
shows the encoding step

\end{description}

\sphinxlineitem{Returns: }\begin{description}
\sphinxlineitem{encoded\_value: BITARRAY}
\sphinxAtStartPar
binary string of the encoded value.

\sphinxlineitem{lenght: int}
\sphinxAtStartPar
length of the message. To specify for decoder.

\end{description}

\end{description}

\begin{sphinxVerbatim}[commandchars=\\\{\}]
\PYG{g+gp}{\PYGZgt{}\PYGZgt{}\PYGZgt{} }\PYG{n}{symbols} \PYG{o}{=} \PYG{p}{[}\PYG{l+s+s1}{\PYGZsq{}}\PYG{l+s+s1}{a}\PYG{l+s+s1}{\PYGZsq{}}\PYG{p}{,} \PYG{l+s+s1}{\PYGZsq{}}\PYG{l+s+s1}{b}\PYG{l+s+s1}{\PYGZsq{}}\PYG{p}{,} \PYG{l+s+s1}{\PYGZsq{}}\PYG{l+s+s1}{c}\PYG{l+s+s1}{\PYGZsq{}}\PYG{p}{]}
\PYG{g+gp}{\PYGZgt{}\PYGZgt{}\PYGZgt{} }\PYG{n}{freq} \PYG{o}{=} \PYG{p}{[}\PYG{l+m+mf}{0.8}\PYG{p}{,} \PYG{l+m+mf}{0.02}\PYG{p}{,} \PYG{l+m+mf}{0.18}\PYG{p}{]}
\PYG{g+gp}{\PYGZgt{}\PYGZgt{}\PYGZgt{} }\PYG{n}{f} \PYG{o}{=} \PYG{n}{ArithmeticCoding}\PYG{p}{(}\PYG{n}{symbols}\PYG{p}{,} \PYG{n}{freq}\PYG{p}{)}
\PYG{g+gp}{\PYGZgt{}\PYGZgt{}\PYGZgt{} }\PYG{n}{encoded\PYGZus{}value}\PYG{p}{,} \PYG{n}{msg\PYGZus{}len} \PYG{o}{=} \PYG{n}{f}\PYG{o}{.}\PYG{n}{encode}\PYG{p}{(}\PYG{l+s+s1}{\PYGZsq{}}\PYG{l+s+s1}{abaca}\PYG{l+s+s1}{\PYGZsq{}}\PYG{p}{)}
\PYG{g+gp}{\PYGZgt{}\PYGZgt{}\PYGZgt{} }\PYG{n+nb}{print}\PYG{p}{(}\PYG{n}{encoded\PYGZus{}value}\PYG{p}{,} \PYG{n}{msg\PYGZus{}len}\PYG{p}{)}
\PYG{g+go}{bitarray(\PYGZsq{}10100110110\PYGZsq{}) 5}
\end{sphinxVerbatim}

\begin{sphinxVerbatim}[commandchars=\\\{\}]
\PYG{g+gp}{\PYGZgt{}\PYGZgt{}\PYGZgt{} }\PYG{n}{f} \PYG{o}{=} \PYG{n}{ArithmeticCoding}\PYG{p}{(}\PYG{n}{symbols} \PYG{o}{=} \PYG{k+kc}{None}\PYG{p}{,} \PYG{n}{frequency}\PYG{o}{=} \PYG{k+kc}{None}\PYG{p}{,} \PYG{n}{message}\PYG{o}{=}\PYG{l+s+s1}{\PYGZsq{}}\PYG{l+s+s1}{hello}\PYG{l+s+s1}{\PYGZsq{}}\PYG{p}{)}
\PYG{g+gp}{\PYGZgt{}\PYGZgt{}\PYGZgt{} }\PYG{n}{encoded\PYGZus{}value}\PYG{p}{,} \PYG{n}{msg\PYGZus{}len} \PYG{o}{=} \PYG{n}{f}\PYG{o}{.}\PYG{n}{encode}\PYG{p}{(}\PYG{n}{show\PYGZus{}steps} \PYG{o}{=} \PYG{k+kc}{True}\PYG{p}{)}
\PYG{g+go}{Encoding Process}
\PYG{g+go}{\PYGZhy{}\PYGZhy{}\PYGZhy{}\PYGZhy{}\PYGZhy{}\PYGZhy{}\PYGZhy{}\PYGZhy{}\PYGZhy{}\PYGZhy{}\PYGZhy{}\PYGZhy{}\PYGZhy{}\PYGZhy{}\PYGZhy{}\PYGZhy{}\PYGZhy{}\PYGZhy{}}
\PYG{g+go}{+\PYGZhy{}\PYGZhy{}\PYGZhy{}\PYGZhy{}\PYGZhy{}\PYGZhy{}\PYGZhy{}\PYGZhy{}+\PYGZhy{}\PYGZhy{}\PYGZhy{}\PYGZhy{}\PYGZhy{}\PYGZhy{}\PYGZhy{}\PYGZhy{}\PYGZhy{}\PYGZhy{}\PYGZhy{}\PYGZhy{}\PYGZhy{}\PYGZhy{}\PYGZhy{}\PYGZhy{}\PYGZhy{}\PYGZhy{}\PYGZhy{}\PYGZhy{}\PYGZhy{}\PYGZhy{}\PYGZhy{}\PYGZhy{}\PYGZhy{}\PYGZhy{}\PYGZhy{}\PYGZhy{}\PYGZhy{}\PYGZhy{}\PYGZhy{}\PYGZhy{}\PYGZhy{}\PYGZhy{}\PYGZhy{}\PYGZhy{}\PYGZhy{}\PYGZhy{}\PYGZhy{}\PYGZhy{}\PYGZhy{}\PYGZhy{}\PYGZhy{}\PYGZhy{}+\PYGZhy{}\PYGZhy{}\PYGZhy{}\PYGZhy{}\PYGZhy{}\PYGZhy{}\PYGZhy{}\PYGZhy{}\PYGZhy{}\PYGZhy{}\PYGZhy{}\PYGZhy{}\PYGZhy{}\PYGZhy{}\PYGZhy{}\PYGZhy{}\PYGZhy{}\PYGZhy{}+}
\PYG{g+go}{| Symbol |                  Interval                  |      Remark      |}
\PYG{g+go}{+\PYGZhy{}\PYGZhy{}\PYGZhy{}\PYGZhy{}\PYGZhy{}\PYGZhy{}\PYGZhy{}\PYGZhy{}+\PYGZhy{}\PYGZhy{}\PYGZhy{}\PYGZhy{}\PYGZhy{}\PYGZhy{}\PYGZhy{}\PYGZhy{}\PYGZhy{}\PYGZhy{}\PYGZhy{}\PYGZhy{}\PYGZhy{}\PYGZhy{}\PYGZhy{}\PYGZhy{}\PYGZhy{}\PYGZhy{}\PYGZhy{}\PYGZhy{}\PYGZhy{}\PYGZhy{}\PYGZhy{}\PYGZhy{}\PYGZhy{}\PYGZhy{}\PYGZhy{}\PYGZhy{}\PYGZhy{}\PYGZhy{}\PYGZhy{}\PYGZhy{}\PYGZhy{}\PYGZhy{}\PYGZhy{}\PYGZhy{}\PYGZhy{}\PYGZhy{}\PYGZhy{}\PYGZhy{}\PYGZhy{}\PYGZhy{}\PYGZhy{}\PYGZhy{}+\PYGZhy{}\PYGZhy{}\PYGZhy{}\PYGZhy{}\PYGZhy{}\PYGZhy{}\PYGZhy{}\PYGZhy{}\PYGZhy{}\PYGZhy{}\PYGZhy{}\PYGZhy{}\PYGZhy{}\PYGZhy{}\PYGZhy{}\PYGZhy{}\PYGZhy{}\PYGZhy{}+}
\PYG{g+go}{|   h    |                  (0, 0.2)                  | Pick Next Symbol |}
\PYG{g+go}{|   he   | (0.04000000000000001, 0.08000000000000002) | Pick Next Symbol |}
\PYG{g+go}{|  hel   | (0.05600000000000001, 0.07200000000000001) | Pick Next Symbol |}
\PYG{g+go}{|  hell  | (0.06240000000000001, 0.06880000000000001) | Pick Next Symbol |}
\PYG{g+go}{| hello  | (0.06752000000000001, 0.06880000000000001) | Pick Next Symbol |}
\PYG{g+go}{|        |            Symbols Encoded = 5             |                  |}
\PYG{g+go}{|        |         Tag = 0.06816000000000001          |                  |}
\PYG{g+go}{|        | Compressed Value = bitarray(\PYGZsq{}00010001011\PYGZsq{}) |                  |}
\PYG{g+go}{+\PYGZhy{}\PYGZhy{}\PYGZhy{}\PYGZhy{}\PYGZhy{}\PYGZhy{}\PYGZhy{}\PYGZhy{}+\PYGZhy{}\PYGZhy{}\PYGZhy{}\PYGZhy{}\PYGZhy{}\PYGZhy{}\PYGZhy{}\PYGZhy{}\PYGZhy{}\PYGZhy{}\PYGZhy{}\PYGZhy{}\PYGZhy{}\PYGZhy{}\PYGZhy{}\PYGZhy{}\PYGZhy{}\PYGZhy{}\PYGZhy{}\PYGZhy{}\PYGZhy{}\PYGZhy{}\PYGZhy{}\PYGZhy{}\PYGZhy{}\PYGZhy{}\PYGZhy{}\PYGZhy{}\PYGZhy{}\PYGZhy{}\PYGZhy{}\PYGZhy{}\PYGZhy{}\PYGZhy{}\PYGZhy{}\PYGZhy{}\PYGZhy{}\PYGZhy{}\PYGZhy{}\PYGZhy{}\PYGZhy{}\PYGZhy{}\PYGZhy{}\PYGZhy{}+\PYGZhy{}\PYGZhy{}\PYGZhy{}\PYGZhy{}\PYGZhy{}\PYGZhy{}\PYGZhy{}\PYGZhy{}\PYGZhy{}\PYGZhy{}\PYGZhy{}\PYGZhy{}\PYGZhy{}\PYGZhy{}\PYGZhy{}\PYGZhy{}\PYGZhy{}\PYGZhy{}+}
\end{sphinxVerbatim}

\end{fulllineitems}

\index{msg\_prob() (arithmetic\_coding.ArithmeticCoding method)@\spxentry{msg\_prob()}\spxextra{arithmetic\_coding.ArithmeticCoding method}}

\begin{fulllineitems}
\phantomsection\label{\detokenize{arithmetic_coding:arithmetic_coding.ArithmeticCoding.msg_prob}}
\pysigstartsignatures
\pysiglinewithargsret{\sphinxbfcode{\sphinxupquote{msg\_prob}}}{\emph{\DUrole{n}{message}\DUrole{p}{:}\DUrole{w}{  }\DUrole{n}{list}}}{{ $\rightarrow$ float}}
\pysigstopsignatures\begin{description}
\sphinxlineitem{Computes the joint probability of message. Requires the class to initianted. }
\sphinxAtStartPar
P(x\_1, x\_2, …) = p(x\_1).p(x\_2). …

\sphinxlineitem{Parameters:}
\sphinxAtStartPar
message: list

\sphinxlineitem{Returns:}\begin{description}
\sphinxlineitem{prob: float}
\sphinxAtStartPar
the probability of the entire message

\end{description}

\end{description}

\begin{sphinxVerbatim}[commandchars=\\\{\}]
\PYG{g+gp}{\PYGZgt{}\PYGZgt{}\PYGZgt{} }\PYG{n}{f} \PYG{o}{=} \PYG{n}{ArithmeticCoding}\PYG{p}{(}\PYG{p}{[}\PYG{l+s+s1}{\PYGZsq{}}\PYG{l+s+s1}{a}\PYG{l+s+s1}{\PYGZsq{}}\PYG{p}{,} \PYG{l+s+s1}{\PYGZsq{}}\PYG{l+s+s1}{b}\PYG{l+s+s1}{\PYGZsq{}}\PYG{p}{,} \PYG{l+s+s1}{\PYGZsq{}}\PYG{l+s+s1}{c}\PYG{l+s+s1}{\PYGZsq{}}\PYG{p}{]}\PYG{p}{,} \PYG{p}{[}\PYG{l+m+mf}{0.8}\PYG{p}{,} \PYG{l+m+mf}{0.02}\PYG{p}{,} \PYG{l+m+mf}{0.18}\PYG{p}{]}\PYG{p}{)}
\PYG{g+gp}{\PYGZgt{}\PYGZgt{}\PYGZgt{} }\PYG{n}{f}\PYG{o}{.}\PYG{n}{msg\PYGZus{}prob}\PYG{p}{(}\PYG{l+s+s1}{\PYGZsq{}}\PYG{l+s+s1}{aabcca}\PYG{l+s+s1}{\PYGZsq{}}\PYG{p}{)}
\PYG{g+go}{0.0003317760000000001}
\end{sphinxVerbatim}

\end{fulllineitems}


\end{fulllineitems}

\index{ArithmeticDecoder (class in arithmetic\_coding)@\spxentry{ArithmeticDecoder}\spxextra{class in arithmetic\_coding}}

\begin{fulllineitems}
\phantomsection\label{\detokenize{arithmetic_coding:arithmetic_coding.ArithmeticDecoder}}
\pysigstartsignatures
\pysiglinewithargsret{\sphinxbfcode{\sphinxupquote{class\DUrole{w}{  }}}\sphinxcode{\sphinxupquote{arithmetic\_coding.}}\sphinxbfcode{\sphinxupquote{ArithmeticDecoder}}}{\emph{\DUrole{n}{symbols}\DUrole{p}{:}\DUrole{w}{  }\DUrole{n}{list}}, \emph{\DUrole{n}{frequency}\DUrole{p}{:}\DUrole{w}{  }\DUrole{n}{list}}}{}
\pysigstopsignatures
\sphinxAtStartPar
Bases: {\hyperref[\detokenize{core:core.data.Data}]{\sphinxcrossref{\sphinxcode{\sphinxupquote{Data}}}}}

\sphinxAtStartPar
Arithmetic decoder class. Used only for decoing. Assume a communication channel where the receiver has access to the decoding channel only. instantiates the arithmetic coding class and uses the decoding function.
\begin{description}
\sphinxlineitem{Attributes:}\begin{description}
\sphinxlineitem{symbols}{[}list{]}
\sphinxAtStartPar
list of symbols, elements can be any format

\sphinxlineitem{frequency: list}
\sphinxAtStartPar
frequency list associated with the list

\end{description}

\end{description}

\begin{sphinxVerbatim}[commandchars=\\\{\}]
\PYG{g+gp}{\PYGZgt{}\PYGZgt{}\PYGZgt{} }\PYG{n}{symbols} \PYG{o}{=} \PYG{p}{[}\PYG{l+s+s1}{\PYGZsq{}}\PYG{l+s+s1}{a}\PYG{l+s+s1}{\PYGZsq{}}\PYG{p}{,} \PYG{l+s+s1}{\PYGZsq{}}\PYG{l+s+s1}{b}\PYG{l+s+s1}{\PYGZsq{}}\PYG{p}{,} \PYG{l+s+s1}{\PYGZsq{}}\PYG{l+s+s1}{c}\PYG{l+s+s1}{\PYGZsq{}}\PYG{p}{]}
\PYG{g+gp}{\PYGZgt{}\PYGZgt{}\PYGZgt{} }\PYG{n}{freq} \PYG{o}{=} \PYG{p}{[}\PYG{l+m+mf}{0.8}\PYG{p}{,} \PYG{l+m+mf}{0.02}\PYG{p}{,} \PYG{l+m+mf}{0.18}\PYG{p}{]}
\PYG{g+gp}{\PYGZgt{}\PYGZgt{}\PYGZgt{} }\PYG{n}{f} \PYG{o}{=} \PYG{n}{ArithmeticDecoder}\PYG{p}{(}\PYG{n}{symbols}\PYG{p}{,} \PYG{n}{freq}\PYG{p}{)}
\PYG{g+gp}{\PYGZgt{}\PYGZgt{}\PYGZgt{} }\PYG{n}{f}\PYG{o}{.}\PYG{n}{decode}\PYG{p}{(}\PYG{n}{bitarray}\PYG{o}{.}\PYG{n}{bitarray}\PYG{p}{(}\PYG{l+s+s1}{\PYGZsq{}}\PYG{l+s+s1}{10100110110}\PYG{l+s+s1}{\PYGZsq{}}\PYG{p}{)}\PYG{p}{,}\PYG{l+m+mi}{5}\PYG{p}{)}
\PYG{g+go}{abaca}
\end{sphinxVerbatim}
\index{decode() (arithmetic\_coding.ArithmeticDecoder method)@\spxentry{decode()}\spxextra{arithmetic\_coding.ArithmeticDecoder method}}

\begin{fulllineitems}
\phantomsection\label{\detokenize{arithmetic_coding:arithmetic_coding.ArithmeticDecoder.decode}}
\pysigstartsignatures
\pysiglinewithargsret{\sphinxbfcode{\sphinxupquote{decode}}}{\emph{\DUrole{n}{encoded\_value}\DUrole{p}{:}\DUrole{w}{  }\DUrole{n}{bitarray.bitarray}}, \emph{\DUrole{n}{msg\_length}\DUrole{p}{:}\DUrole{w}{  }\DUrole{n}{int}}}{}
\pysigstopsignatures
\sphinxAtStartPar
Decodes a bit array using airithmetic coding scheme. 
Parameters:
\begin{quote}
\begin{description}
\sphinxlineitem{encoded\_value: bitarray.bitarray}
\sphinxAtStartPar
bitarray instance of the encoded value.

\sphinxlineitem{msg\_length: int}
\sphinxAtStartPar
length of the message. needs to be specified to the same number as original msg to get right decoding.

\end{description}
\end{quote}
\begin{description}
\sphinxlineitem{Returns:}\begin{description}
\sphinxlineitem{decoded\_symbols: str}
\sphinxAtStartPar
returns the decoded symbols

\end{description}

\end{description}

\end{fulllineitems}


\end{fulllineitems}

\index{RangeCoding (class in arithmetic\_coding)@\spxentry{RangeCoding}\spxextra{class in arithmetic\_coding}}

\begin{fulllineitems}
\phantomsection\label{\detokenize{arithmetic_coding:arithmetic_coding.RangeCoding}}
\pysigstartsignatures
\pysiglinewithargsret{\sphinxbfcode{\sphinxupquote{class\DUrole{w}{  }}}\sphinxcode{\sphinxupquote{arithmetic\_coding.}}\sphinxbfcode{\sphinxupquote{RangeCoding}}}{\emph{\DUrole{n}{symbols}\DUrole{p}{:}\DUrole{w}{  }\DUrole{n}{list}}, \emph{\DUrole{n}{frequency}\DUrole{p}{:}\DUrole{w}{  }\DUrole{n}{list}}, \emph{\DUrole{n}{message}\DUrole{p}{:}\DUrole{w}{  }\DUrole{n}{str}\DUrole{w}{  }\DUrole{o}{=}\DUrole{w}{  }\DUrole{default_value}{None}}}{}
\pysigstopsignatures
\sphinxAtStartPar
Bases: {\hyperref[\detokenize{arithmetic_coding:arithmetic_coding.ArithmeticCoding}]{\sphinxcrossref{\sphinxcode{\sphinxupquote{ArithmeticCoding}}}}}

\sphinxAtStartPar
Class for arithmetic coding compression with rescaling. Might not be efficient. Inherits the arithmetic coding class and changes teh encoding and decodign function. Allows compression in two ways.
\begin{enumerate}
\sphinxsetlistlabels{\arabic}{enumi}{enumii}{}{.}%
\item {} \begin{description}
\sphinxlineitem{By specifying the symbols and frequency.}
\sphinxAtStartPar
In this case arg:msg must be provided in the encode step.

\end{description}

\item {} \begin{description}
\sphinxlineitem{By giving the entire message itself. }
\sphinxAtStartPar
No argument required in the encode step. 
Computes the probability and cumulative distribution from the message itself.

\end{description}

\end{enumerate}
\begin{description}
\sphinxlineitem{Attributes:}\begin{description}
\sphinxlineitem{symbols}{[}list{]}
\sphinxAtStartPar
list of symbols, elements can be any format

\sphinxlineitem{frequency: list}
\sphinxAtStartPar
frequency list associated with the list

\sphinxlineitem{message: list, default = None}
\sphinxAtStartPar
list of message.

\end{description}

\end{description}
\index{decode() (arithmetic\_coding.RangeCoding method)@\spxentry{decode()}\spxextra{arithmetic\_coding.RangeCoding method}}

\begin{fulllineitems}
\phantomsection\label{\detokenize{arithmetic_coding:arithmetic_coding.RangeCoding.decode}}
\pysigstartsignatures
\pysiglinewithargsret{\sphinxbfcode{\sphinxupquote{decode}}}{\emph{\DUrole{n}{encoded\_value}\DUrole{p}{:}\DUrole{w}{  }\DUrole{n}{bitarray.bitarray}}, \emph{\DUrole{n}{msg\_length}\DUrole{p}{:}\DUrole{w}{  }\DUrole{n}{int}}, \emph{\DUrole{n}{show\_steps}\DUrole{p}{:}\DUrole{w}{  }\DUrole{n}{bool}\DUrole{w}{  }\DUrole{o}{=}\DUrole{w}{  }\DUrole{default_value}{False}}}{}
\pysigstopsignatures
\sphinxAtStartPar
Using the decoding by checking interval, rescaling, updating interval, and picking new symbol.
\begin{description}
\sphinxlineitem{Parameters: }\begin{description}
\sphinxlineitem{encoded\_value: bitarray.bitarray}
\sphinxAtStartPar
bitarray instance of the encoded value.

\sphinxlineitem{msg\_length: int}
\sphinxAtStartPar
length of the message. needs to be specified to the same number as original msg to get right decoding.

\sphinxlineitem{show\_steps: bool, default = False}
\sphinxAtStartPar
shows decoding steps.

\end{description}

\sphinxlineitem{Returns:}\begin{description}
\sphinxlineitem{decoded\_symbols: str}
\sphinxAtStartPar
returns the decoded symbols

\end{description}

\end{description}

\begin{sphinxVerbatim}[commandchars=\\\{\}]
\PYG{g+gp}{\PYGZgt{}\PYGZgt{}\PYGZgt{} }\PYG{n}{symbols} \PYG{o}{=} \PYG{p}{[}\PYG{l+s+s1}{\PYGZsq{}}\PYG{l+s+s1}{a}\PYG{l+s+s1}{\PYGZsq{}}\PYG{p}{,} \PYG{l+s+s1}{\PYGZsq{}}\PYG{l+s+s1}{b}\PYG{l+s+s1}{\PYGZsq{}}\PYG{p}{,} \PYG{l+s+s1}{\PYGZsq{}}\PYG{l+s+s1}{c}\PYG{l+s+s1}{\PYGZsq{}}\PYG{p}{]}
\PYG{g+gp}{\PYGZgt{}\PYGZgt{}\PYGZgt{} }\PYG{n}{freq} \PYG{o}{=} \PYG{p}{[}\PYG{l+m+mf}{0.8}\PYG{p}{,} \PYG{l+m+mf}{0.02}\PYG{p}{,} \PYG{l+m+mf}{0.18}\PYG{p}{]}
\PYG{g+gp}{\PYGZgt{}\PYGZgt{}\PYGZgt{} }\PYG{n}{f} \PYG{o}{=} \PYG{n}{RangeCoding}\PYG{p}{(}\PYG{n}{symbols}\PYG{p}{,} \PYG{n}{freq}\PYG{p}{)}
\PYG{g+gp}{\PYGZgt{}\PYGZgt{}\PYGZgt{} }\PYG{n}{decoded\PYGZus{}value} \PYG{o}{=} \PYG{n}{f}\PYG{o}{.}\PYG{n}{decode}\PYG{p}{(}\PYG{n}{bitarray}\PYG{o}{.}\PYG{n}{bitarray}\PYG{p}{(}\PYG{l+s+s1}{\PYGZsq{}}\PYG{l+s+s1}{1010011011}\PYG{l+s+s1}{\PYGZsq{}}\PYG{p}{)}\PYG{p}{,} \PYG{l+m+mi}{5}\PYG{p}{)}
\PYG{g+gp}{\PYGZgt{}\PYGZgt{}\PYGZgt{} }\PYG{n+nb}{print}\PYG{p}{(}\PYG{n}{decoded\PYGZus{}value}\PYG{p}{)}
\PYG{g+go}{abaca}
\end{sphinxVerbatim}

\begin{sphinxVerbatim}[commandchars=\\\{\}]
\PYG{g+gp}{\PYGZgt{}\PYGZgt{}\PYGZgt{} }\PYG{n}{decoded\PYGZus{}value} \PYG{o}{=} \PYG{n}{f}\PYG{o}{.}\PYG{n}{decode}\PYG{p}{(}\PYG{n}{encoded\PYGZus{}value}\PYG{p}{,} \PYG{n}{msg\PYGZus{}len}\PYG{p}{,} \PYG{n}{show\PYGZus{}steps}\PYG{o}{=}\PYG{k+kc}{True}\PYG{p}{)}
\PYG{g+go}{Decoding Process}
\PYG{g+go}{\PYGZhy{}\PYGZhy{}\PYGZhy{}\PYGZhy{}\PYGZhy{}\PYGZhy{}\PYGZhy{}\PYGZhy{}\PYGZhy{}\PYGZhy{}\PYGZhy{}\PYGZhy{}\PYGZhy{}\PYGZhy{}\PYGZhy{}\PYGZhy{}\PYGZhy{}\PYGZhy{}}
\PYG{g+go}{+\PYGZhy{}\PYGZhy{}\PYGZhy{}\PYGZhy{}\PYGZhy{}\PYGZhy{}\PYGZhy{}\PYGZhy{}\PYGZhy{}\PYGZhy{}\PYGZhy{}\PYGZhy{}\PYGZhy{}\PYGZhy{}+\PYGZhy{}\PYGZhy{}\PYGZhy{}\PYGZhy{}\PYGZhy{}\PYGZhy{}\PYGZhy{}\PYGZhy{}\PYGZhy{}\PYGZhy{}\PYGZhy{}\PYGZhy{}\PYGZhy{}\PYGZhy{}\PYGZhy{}\PYGZhy{}\PYGZhy{}\PYGZhy{}\PYGZhy{}\PYGZhy{}\PYGZhy{}\PYGZhy{}\PYGZhy{}+\PYGZhy{}\PYGZhy{}\PYGZhy{}\PYGZhy{}\PYGZhy{}\PYGZhy{}\PYGZhy{}\PYGZhy{}\PYGZhy{}\PYGZhy{}\PYGZhy{}\PYGZhy{}\PYGZhy{}+\PYGZhy{}\PYGZhy{}\PYGZhy{}\PYGZhy{}\PYGZhy{}\PYGZhy{}\PYGZhy{}\PYGZhy{}\PYGZhy{}\PYGZhy{}\PYGZhy{}\PYGZhy{}\PYGZhy{}\PYGZhy{}\PYGZhy{}\PYGZhy{}\PYGZhy{}\PYGZhy{}\PYGZhy{}\PYGZhy{}\PYGZhy{}\PYGZhy{}\PYGZhy{}\PYGZhy{}\PYGZhy{}\PYGZhy{}\PYGZhy{}\PYGZhy{}\PYGZhy{}\PYGZhy{}\PYGZhy{}\PYGZhy{}\PYGZhy{}\PYGZhy{}\PYGZhy{}\PYGZhy{}\PYGZhy{}\PYGZhy{}\PYGZhy{}\PYGZhy{}\PYGZhy{}\PYGZhy{}\PYGZhy{}\PYGZhy{}+\PYGZhy{}\PYGZhy{}\PYGZhy{}\PYGZhy{}\PYGZhy{}\PYGZhy{}\PYGZhy{}\PYGZhy{}\PYGZhy{}\PYGZhy{}\PYGZhy{}\PYGZhy{}\PYGZhy{}\PYGZhy{}\PYGZhy{}+}
\PYG{g+go}{| Decoded Symb |     Encoded Value     |     Tag     |                   Range                    |    Remark     |}
\PYG{g+go}{+\PYGZhy{}\PYGZhy{}\PYGZhy{}\PYGZhy{}\PYGZhy{}\PYGZhy{}\PYGZhy{}\PYGZhy{}\PYGZhy{}\PYGZhy{}\PYGZhy{}\PYGZhy{}\PYGZhy{}\PYGZhy{}+\PYGZhy{}\PYGZhy{}\PYGZhy{}\PYGZhy{}\PYGZhy{}\PYGZhy{}\PYGZhy{}\PYGZhy{}\PYGZhy{}\PYGZhy{}\PYGZhy{}\PYGZhy{}\PYGZhy{}\PYGZhy{}\PYGZhy{}\PYGZhy{}\PYGZhy{}\PYGZhy{}\PYGZhy{}\PYGZhy{}\PYGZhy{}\PYGZhy{}\PYGZhy{}+\PYGZhy{}\PYGZhy{}\PYGZhy{}\PYGZhy{}\PYGZhy{}\PYGZhy{}\PYGZhy{}\PYGZhy{}\PYGZhy{}\PYGZhy{}\PYGZhy{}\PYGZhy{}\PYGZhy{}+\PYGZhy{}\PYGZhy{}\PYGZhy{}\PYGZhy{}\PYGZhy{}\PYGZhy{}\PYGZhy{}\PYGZhy{}\PYGZhy{}\PYGZhy{}\PYGZhy{}\PYGZhy{}\PYGZhy{}\PYGZhy{}\PYGZhy{}\PYGZhy{}\PYGZhy{}\PYGZhy{}\PYGZhy{}\PYGZhy{}\PYGZhy{}\PYGZhy{}\PYGZhy{}\PYGZhy{}\PYGZhy{}\PYGZhy{}\PYGZhy{}\PYGZhy{}\PYGZhy{}\PYGZhy{}\PYGZhy{}\PYGZhy{}\PYGZhy{}\PYGZhy{}\PYGZhy{}\PYGZhy{}\PYGZhy{}\PYGZhy{}\PYGZhy{}\PYGZhy{}\PYGZhy{}\PYGZhy{}\PYGZhy{}\PYGZhy{}+\PYGZhy{}\PYGZhy{}\PYGZhy{}\PYGZhy{}\PYGZhy{}\PYGZhy{}\PYGZhy{}\PYGZhy{}\PYGZhy{}\PYGZhy{}\PYGZhy{}\PYGZhy{}\PYGZhy{}\PYGZhy{}\PYGZhy{}+}
\PYG{g+go}{|      h       | bitarray(\PYGZsq{}000100011\PYGZsq{}) | 0.068359375 |                  (0, 0.2)                  | Left Scaling  |}
\PYG{g+go}{|              |                       |             |                                            |   Remove 0    |}
\PYG{g+go}{|      h       | bitarray(\PYGZsq{}00100011\PYGZsq{})  | 0.13671875  |                  (0, 0.4)                  | Left Scaling  |}
\PYG{g+go}{|              |                       |             |                                            |   Remove 0    |}
\PYG{g+go}{|      h       |  bitarray(\PYGZsq{}0100011\PYGZsq{})  |  0.2734375  |                  (0, 0.8)                  |   Pick next   |}
\PYG{g+go}{|      he      |  bitarray(\PYGZsq{}0100011\PYGZsq{})  |  0.2734375  | (0.16000000000000003, 0.32000000000000006) | Left Scaling  |}
\PYG{g+go}{|              |                       |             |                                            |   Remove 0    |}
\PYG{g+go}{|      he      |  bitarray(\PYGZsq{}100011\PYGZsq{})   |  0.546875   | (0.32000000000000006, 0.6400000000000001)  |   Pick next   |}
\PYG{g+go}{|     hel      |  bitarray(\PYGZsq{}100011\PYGZsq{})   |  0.546875   | (0.44800000000000006, 0.5760000000000001)  |   Pick next   |}
\PYG{g+go}{|     hell     |  bitarray(\PYGZsq{}100011\PYGZsq{})   |  0.546875   |  (0.4992000000000001, 0.5504000000000001)  |   Pick next   |}
\PYG{g+go}{|    hello     |  bitarray(\PYGZsq{}100011\PYGZsq{})   |  0.546875   |  (0.5401600000000001, 0.5504000000000001)  | Right Scaling |}
\PYG{g+go}{|              |                       |             |                                            |   Remove 1    |}
\PYG{g+go}{|    hello     |   bitarray(\PYGZsq{}00011\PYGZsq{})   |   0.09375   | (0.08032000000000017, 0.10080000000000022) | Left Scaling  |}
\PYG{g+go}{|              |                       |             |                                            |   Remove 0    |}
\PYG{g+go}{|    hello     |   bitarray(\PYGZsq{}0011\PYGZsq{})    |   0.1875    | (0.16064000000000034, 0.20160000000000045) | Left Scaling  |}
\PYG{g+go}{|              |                       |             |                                            |   Remove 0    |}
\PYG{g+go}{|    hello     |    bitarray(\PYGZsq{}011\PYGZsq{})    |    0.375    |  (0.3212800000000007, 0.4032000000000009)  | Left Scaling  |}
\PYG{g+go}{|              |                       |             |                                            |   Remove 0    |}
\PYG{g+go}{|    hello     |    bitarray(\PYGZsq{}11\PYGZsq{})     |    0.75     |  (0.6425600000000014, 0.8064000000000018)  | Right Scaling |}
\PYG{g+go}{|              |                       |             |                                            |   Remove 1    |}
\PYG{g+go}{|    hello     |     bitarray(\PYGZsq{}1\PYGZsq{})     |     0.5     |  (0.2851200000000027, 0.6128000000000036)  |   Pick next   |}
\PYG{g+go}{+\PYGZhy{}\PYGZhy{}\PYGZhy{}\PYGZhy{}\PYGZhy{}\PYGZhy{}\PYGZhy{}\PYGZhy{}\PYGZhy{}\PYGZhy{}\PYGZhy{}\PYGZhy{}\PYGZhy{}\PYGZhy{}+\PYGZhy{}\PYGZhy{}\PYGZhy{}\PYGZhy{}\PYGZhy{}\PYGZhy{}\PYGZhy{}\PYGZhy{}\PYGZhy{}\PYGZhy{}\PYGZhy{}\PYGZhy{}\PYGZhy{}\PYGZhy{}\PYGZhy{}\PYGZhy{}\PYGZhy{}\PYGZhy{}\PYGZhy{}\PYGZhy{}\PYGZhy{}\PYGZhy{}\PYGZhy{}+\PYGZhy{}\PYGZhy{}\PYGZhy{}\PYGZhy{}\PYGZhy{}\PYGZhy{}\PYGZhy{}\PYGZhy{}\PYGZhy{}\PYGZhy{}\PYGZhy{}\PYGZhy{}\PYGZhy{}+\PYGZhy{}\PYGZhy{}\PYGZhy{}\PYGZhy{}\PYGZhy{}\PYGZhy{}\PYGZhy{}\PYGZhy{}\PYGZhy{}\PYGZhy{}\PYGZhy{}\PYGZhy{}\PYGZhy{}\PYGZhy{}\PYGZhy{}\PYGZhy{}\PYGZhy{}\PYGZhy{}\PYGZhy{}\PYGZhy{}\PYGZhy{}\PYGZhy{}\PYGZhy{}\PYGZhy{}\PYGZhy{}\PYGZhy{}\PYGZhy{}\PYGZhy{}\PYGZhy{}\PYGZhy{}\PYGZhy{}\PYGZhy{}\PYGZhy{}\PYGZhy{}\PYGZhy{}\PYGZhy{}\PYGZhy{}\PYGZhy{}\PYGZhy{}\PYGZhy{}\PYGZhy{}\PYGZhy{}\PYGZhy{}\PYGZhy{}+\PYGZhy{}\PYGZhy{}\PYGZhy{}\PYGZhy{}\PYGZhy{}\PYGZhy{}\PYGZhy{}\PYGZhy{}\PYGZhy{}\PYGZhy{}\PYGZhy{}\PYGZhy{}\PYGZhy{}\PYGZhy{}\PYGZhy{}+}
\PYG{g+go}{Decoded Value = hello}
\end{sphinxVerbatim}

\end{fulllineitems}

\index{encode() (arithmetic\_coding.RangeCoding method)@\spxentry{encode()}\spxextra{arithmetic\_coding.RangeCoding method}}

\begin{fulllineitems}
\phantomsection\label{\detokenize{arithmetic_coding:arithmetic_coding.RangeCoding.encode}}
\pysigstartsignatures
\pysiglinewithargsret{\sphinxbfcode{\sphinxupquote{encode}}}{\emph{\DUrole{n}{msg}\DUrole{p}{:}\DUrole{w}{  }\DUrole{n}{list}\DUrole{w}{  }\DUrole{o}{=}\DUrole{w}{  }\DUrole{default_value}{None}}, \emph{\DUrole{n}{show\_steps}\DUrole{p}{:}\DUrole{w}{  }\DUrole{n}{bool}\DUrole{w}{  }\DUrole{o}{=}\DUrole{w}{  }\DUrole{default_value}{False}}}{{ $\rightarrow$ Tuple\DUrole{p}{{[}}bitarray.bitarray\DUrole{p}{,}\DUrole{w}{  }int\DUrole{p}{{]}}}}
\pysigstopsignatures
\sphinxAtStartPar
Range encoding function. Can be used to encdoe in either ways as specifiec before.
\begin{description}
\sphinxlineitem{Parameters:}\begin{description}
\sphinxlineitem{msg: list, default = None}
\sphinxAtStartPar
message you want to encode

\sphinxlineitem{show\_steps: bool, defalut = False}
\sphinxAtStartPar
shows encoding step

\end{description}

\sphinxlineitem{Returns: }\begin{description}
\sphinxlineitem{encoded\_value: BITARRAY}
\sphinxAtStartPar
binary string of the encoded value.

\sphinxlineitem{lenght: int}
\sphinxAtStartPar
length of the message. To specify for decoder.

\end{description}

\end{description}

\begin{sphinxVerbatim}[commandchars=\\\{\}]
\PYG{g+gp}{\PYGZgt{}\PYGZgt{}\PYGZgt{} }\PYG{n}{symbols} \PYG{o}{=} \PYG{p}{[}\PYG{l+s+s1}{\PYGZsq{}}\PYG{l+s+s1}{a}\PYG{l+s+s1}{\PYGZsq{}}\PYG{p}{,} \PYG{l+s+s1}{\PYGZsq{}}\PYG{l+s+s1}{b}\PYG{l+s+s1}{\PYGZsq{}}\PYG{p}{,} \PYG{l+s+s1}{\PYGZsq{}}\PYG{l+s+s1}{c}\PYG{l+s+s1}{\PYGZsq{}}\PYG{p}{]}
\PYG{g+gp}{\PYGZgt{}\PYGZgt{}\PYGZgt{} }\PYG{n}{freq} \PYG{o}{=} \PYG{p}{[}\PYG{l+m+mf}{0.8}\PYG{p}{,} \PYG{l+m+mf}{0.02}\PYG{p}{,} \PYG{l+m+mf}{0.18}\PYG{p}{]}
\PYG{g+gp}{\PYGZgt{}\PYGZgt{}\PYGZgt{} }\PYG{n}{f} \PYG{o}{=} \PYG{n}{RangeCoding}\PYG{p}{(}\PYG{n}{symbols}\PYG{p}{,} \PYG{n}{freq}\PYG{p}{)}
\PYG{g+gp}{\PYGZgt{}\PYGZgt{}\PYGZgt{} }\PYG{n}{encoded\PYGZus{}value}\PYG{p}{,} \PYG{n}{msg\PYGZus{}len} \PYG{o}{=} \PYG{n}{f}\PYG{o}{.}\PYG{n}{encode}\PYG{p}{(}\PYG{l+s+s1}{\PYGZsq{}}\PYG{l+s+s1}{abaca}\PYG{l+s+s1}{\PYGZsq{}}\PYG{p}{)}
\PYG{g+gp}{\PYGZgt{}\PYGZgt{}\PYGZgt{} }\PYG{n+nb}{print}\PYG{p}{(}\PYG{n}{encoded\PYGZus{}value}\PYG{p}{,} \PYG{n}{msg\PYGZus{}len}\PYG{p}{)}
\PYG{g+go}{bitarray(\PYGZsq{}1010011011\PYGZsq{}) 5}
\end{sphinxVerbatim}

\begin{sphinxVerbatim}[commandchars=\\\{\}]
\PYG{g+gp}{\PYGZgt{}\PYGZgt{}\PYGZgt{} }\PYG{n}{f} \PYG{o}{=} \PYG{n}{RangeCoding}\PYG{p}{(}\PYG{n}{symbols} \PYG{o}{=} \PYG{k+kc}{None}\PYG{p}{,} \PYG{n}{frequency}\PYG{o}{=} \PYG{k+kc}{None}\PYG{p}{,} \PYG{n}{message}\PYG{o}{=}\PYG{l+s+s1}{\PYGZsq{}}\PYG{l+s+s1}{hello}\PYG{l+s+s1}{\PYGZsq{}}\PYG{p}{,} \PYG{n}{show\PYGZus{}steps}\PYG{o}{=}\PYG{k+kc}{True}\PYG{p}{)}
\PYG{g+gp}{\PYGZgt{}\PYGZgt{}\PYGZgt{} }\PYG{n}{encoded\PYGZus{}value}\PYG{p}{,} \PYG{n}{msg\PYGZus{}len} \PYG{o}{=} \PYG{n}{f}\PYG{o}{.}\PYG{n}{encode}\PYG{p}{(}\PYG{p}{)}
\PYG{g+go}{Encoding Process}
\PYG{g+go}{\PYGZhy{}\PYGZhy{}\PYGZhy{}\PYGZhy{}\PYGZhy{}\PYGZhy{}\PYGZhy{}\PYGZhy{}\PYGZhy{}\PYGZhy{}\PYGZhy{}\PYGZhy{}\PYGZhy{}\PYGZhy{}\PYGZhy{}\PYGZhy{}\PYGZhy{}\PYGZhy{}}
\PYG{g+go}{+\PYGZhy{}\PYGZhy{}\PYGZhy{}\PYGZhy{}\PYGZhy{}\PYGZhy{}\PYGZhy{}\PYGZhy{}+\PYGZhy{}\PYGZhy{}\PYGZhy{}\PYGZhy{}\PYGZhy{}\PYGZhy{}\PYGZhy{}\PYGZhy{}\PYGZhy{}\PYGZhy{}\PYGZhy{}\PYGZhy{}\PYGZhy{}\PYGZhy{}\PYGZhy{}\PYGZhy{}\PYGZhy{}\PYGZhy{}\PYGZhy{}\PYGZhy{}\PYGZhy{}\PYGZhy{}\PYGZhy{}\PYGZhy{}\PYGZhy{}\PYGZhy{}\PYGZhy{}\PYGZhy{}\PYGZhy{}\PYGZhy{}\PYGZhy{}\PYGZhy{}\PYGZhy{}\PYGZhy{}\PYGZhy{}\PYGZhy{}\PYGZhy{}\PYGZhy{}\PYGZhy{}\PYGZhy{}\PYGZhy{}\PYGZhy{}\PYGZhy{}\PYGZhy{}+\PYGZhy{}\PYGZhy{}\PYGZhy{}\PYGZhy{}\PYGZhy{}\PYGZhy{}\PYGZhy{}\PYGZhy{}\PYGZhy{}\PYGZhy{}\PYGZhy{}\PYGZhy{}\PYGZhy{}\PYGZhy{}\PYGZhy{}\PYGZhy{}\PYGZhy{}\PYGZhy{}+}
\PYG{g+go}{| Symbol |                  Interval                  |      Remark      |}
\PYG{g+go}{+\PYGZhy{}\PYGZhy{}\PYGZhy{}\PYGZhy{}\PYGZhy{}\PYGZhy{}\PYGZhy{}\PYGZhy{}+\PYGZhy{}\PYGZhy{}\PYGZhy{}\PYGZhy{}\PYGZhy{}\PYGZhy{}\PYGZhy{}\PYGZhy{}\PYGZhy{}\PYGZhy{}\PYGZhy{}\PYGZhy{}\PYGZhy{}\PYGZhy{}\PYGZhy{}\PYGZhy{}\PYGZhy{}\PYGZhy{}\PYGZhy{}\PYGZhy{}\PYGZhy{}\PYGZhy{}\PYGZhy{}\PYGZhy{}\PYGZhy{}\PYGZhy{}\PYGZhy{}\PYGZhy{}\PYGZhy{}\PYGZhy{}\PYGZhy{}\PYGZhy{}\PYGZhy{}\PYGZhy{}\PYGZhy{}\PYGZhy{}\PYGZhy{}\PYGZhy{}\PYGZhy{}\PYGZhy{}\PYGZhy{}\PYGZhy{}\PYGZhy{}\PYGZhy{}+\PYGZhy{}\PYGZhy{}\PYGZhy{}\PYGZhy{}\PYGZhy{}\PYGZhy{}\PYGZhy{}\PYGZhy{}\PYGZhy{}\PYGZhy{}\PYGZhy{}\PYGZhy{}\PYGZhy{}\PYGZhy{}\PYGZhy{}\PYGZhy{}\PYGZhy{}\PYGZhy{}+}
\PYG{g+go}{|   h    |                  (0, 0.2)                  |  Left Scaling    |}
\PYG{g+go}{|        |                                            |    Output = 0    |}
\PYG{g+go}{|   h    |                  (0, 0.4)                  |  Left Scaling    |}
\PYG{g+go}{|        |                                            |    Output = 0    |}
\PYG{g+go}{|   h    |                  (0, 0.8)                  | Pick Next Symbol |}
\PYG{g+go}{|   he   | (0.16000000000000003, 0.32000000000000006) |  Left Scaling    |}
\PYG{g+go}{|        |                                            |    Output = 0    |}
\PYG{g+go}{|   he   | (0.32000000000000006, 0.6400000000000001)  | Pick Next Symbol |}
\PYG{g+go}{|  hel   | (0.44800000000000006, 0.5760000000000001)  | Pick Next Symbol |}
\PYG{g+go}{|  hell  |  (0.4992000000000001, 0.5504000000000001)  | Pick Next Symbol |}
\PYG{g+go}{| hello  |  (0.5401600000000001, 0.5504000000000001)  |  Right Scaling   |}
\PYG{g+go}{|        |                                            |    Output = 1    |}
\PYG{g+go}{| hello  | (0.08032000000000017, 0.10080000000000022) |  Left Scaling    |}
\PYG{g+go}{|        |                                            |    Output = 0    |}
\PYG{g+go}{| hello  | (0.16064000000000034, 0.20160000000000045) |  Left Scaling    |}
\PYG{g+go}{|        |                                            |    Output = 0    |}
\PYG{g+go}{| hello  |  (0.3212800000000007, 0.4032000000000009)  |  Left Scaling    |}
\PYG{g+go}{|        |                                            |    Output = 0    |}
\PYG{g+go}{| hello  |  (0.6425600000000014, 0.8064000000000018)  |  Right Scaling   |}
\PYG{g+go}{|        |                                            |    Output = 1    |}
\PYG{g+go}{| hello  |  (0.2851200000000027, 0.6128000000000036)  | Pick Next Symbol |}
\PYG{g+go}{|        |            Symbols Encoded = 5             |                  |}
\PYG{g+go}{|        |  Rescaling Output = bitarray(\PYGZsq{}000100011\PYGZsq{})  |                  |}
\PYG{g+go}{|        |                 Tag = 0.5                  |                  |}
\PYG{g+go}{|        |  Compressed Value = bitarray(\PYGZsq{}000100011\PYGZsq{})  |                  |}
\PYG{g+go}{+\PYGZhy{}\PYGZhy{}\PYGZhy{}\PYGZhy{}\PYGZhy{}\PYGZhy{}\PYGZhy{}\PYGZhy{}+\PYGZhy{}\PYGZhy{}\PYGZhy{}\PYGZhy{}\PYGZhy{}\PYGZhy{}\PYGZhy{}\PYGZhy{}\PYGZhy{}\PYGZhy{}\PYGZhy{}\PYGZhy{}\PYGZhy{}\PYGZhy{}\PYGZhy{}\PYGZhy{}\PYGZhy{}\PYGZhy{}\PYGZhy{}\PYGZhy{}\PYGZhy{}\PYGZhy{}\PYGZhy{}\PYGZhy{}\PYGZhy{}\PYGZhy{}\PYGZhy{}\PYGZhy{}\PYGZhy{}\PYGZhy{}\PYGZhy{}\PYGZhy{}\PYGZhy{}\PYGZhy{}\PYGZhy{}\PYGZhy{}\PYGZhy{}\PYGZhy{}\PYGZhy{}\PYGZhy{}\PYGZhy{}\PYGZhy{}\PYGZhy{}\PYGZhy{}+\PYGZhy{}\PYGZhy{}\PYGZhy{}\PYGZhy{}\PYGZhy{}\PYGZhy{}\PYGZhy{}\PYGZhy{}\PYGZhy{}\PYGZhy{}\PYGZhy{}\PYGZhy{}\PYGZhy{}\PYGZhy{}\PYGZhy{}\PYGZhy{}\PYGZhy{}\PYGZhy{}+}
\end{sphinxVerbatim}

\end{fulllineitems}


\end{fulllineitems}

\index{RangeDecoder (class in arithmetic\_coding)@\spxentry{RangeDecoder}\spxextra{class in arithmetic\_coding}}

\begin{fulllineitems}
\phantomsection\label{\detokenize{arithmetic_coding:arithmetic_coding.RangeDecoder}}
\pysigstartsignatures
\pysiglinewithargsret{\sphinxbfcode{\sphinxupquote{class\DUrole{w}{  }}}\sphinxcode{\sphinxupquote{arithmetic\_coding.}}\sphinxbfcode{\sphinxupquote{RangeDecoder}}}{\emph{\DUrole{n}{symbols}\DUrole{p}{:}\DUrole{w}{  }\DUrole{n}{list}}, \emph{\DUrole{n}{frequency}\DUrole{p}{:}\DUrole{w}{  }\DUrole{n}{list}}}{}
\pysigstopsignatures
\sphinxAtStartPar
Bases: \sphinxcode{\sphinxupquote{object}}

\sphinxAtStartPar
Range decoder class. Used only for decoing. Assume a communication channel where the receiver has access to the decoding channel only. instantiates the range coding class and uses the decoding function.
\begin{description}
\sphinxlineitem{Attributes:}\begin{description}
\sphinxlineitem{symbols}{[}list{]}
\sphinxAtStartPar
list of symbols, elements can be any format

\sphinxlineitem{frequency: list}
\sphinxAtStartPar
frequency list associated with the list

\end{description}

\end{description}

\begin{sphinxVerbatim}[commandchars=\\\{\}]
\PYG{g+gp}{\PYGZgt{}\PYGZgt{}\PYGZgt{} }\PYG{n}{symbols} \PYG{o}{=} \PYG{p}{[}\PYG{l+s+s1}{\PYGZsq{}}\PYG{l+s+s1}{a}\PYG{l+s+s1}{\PYGZsq{}}\PYG{p}{,} \PYG{l+s+s1}{\PYGZsq{}}\PYG{l+s+s1}{b}\PYG{l+s+s1}{\PYGZsq{}}\PYG{p}{,} \PYG{l+s+s1}{\PYGZsq{}}\PYG{l+s+s1}{c}\PYG{l+s+s1}{\PYGZsq{}}\PYG{p}{]}
\PYG{g+gp}{\PYGZgt{}\PYGZgt{}\PYGZgt{} }\PYG{n}{freq} \PYG{o}{=} \PYG{p}{[}\PYG{l+m+mf}{0.8}\PYG{p}{,} \PYG{l+m+mf}{0.02}\PYG{p}{,} \PYG{l+m+mf}{0.18}\PYG{p}{]}
\PYG{g+gp}{\PYGZgt{}\PYGZgt{}\PYGZgt{} }\PYG{n}{f} \PYG{o}{=} \PYG{n}{RangeDecoder}\PYG{p}{(}\PYG{n}{symbols}\PYG{p}{,} \PYG{n}{freq}\PYG{p}{)}
\PYG{g+gp}{\PYGZgt{}\PYGZgt{}\PYGZgt{} }\PYG{n}{f}\PYG{o}{.}\PYG{n}{decode}\PYG{p}{(}\PYG{n}{bitarray}\PYG{o}{.}\PYG{n}{bitarray}\PYG{p}{(}\PYG{l+s+s1}{\PYGZsq{}}\PYG{l+s+s1}{1010011011}\PYG{l+s+s1}{\PYGZsq{}}\PYG{p}{)}\PYG{p}{,}\PYG{l+m+mi}{5}\PYG{p}{)}
\PYG{g+go}{abaca}
\end{sphinxVerbatim}
\index{decode() (arithmetic\_coding.RangeDecoder method)@\spxentry{decode()}\spxextra{arithmetic\_coding.RangeDecoder method}}

\begin{fulllineitems}
\phantomsection\label{\detokenize{arithmetic_coding:arithmetic_coding.RangeDecoder.decode}}
\pysigstartsignatures
\pysiglinewithargsret{\sphinxbfcode{\sphinxupquote{decode}}}{\emph{\DUrole{n}{encoded\_value}\DUrole{p}{:}\DUrole{w}{  }\DUrole{n}{bitarray.bitarray}}, \emph{\DUrole{n}{msg\_length}\DUrole{p}{:}\DUrole{w}{  }\DUrole{n}{int}}}{}
\pysigstopsignatures
\sphinxAtStartPar
Decodes a bit array using airithmetic coding scheme. 
Parameters:
\begin{quote}
\begin{description}
\sphinxlineitem{encoded\_value: bitarray.bitarray}
\sphinxAtStartPar
bitarray instance of the encoded value.

\sphinxlineitem{msg\_length: int}
\sphinxAtStartPar
length of the message. needs to be specified to the same number as original msg to get right decoding.

\end{description}
\end{quote}
\begin{description}
\sphinxlineitem{Returns:}\begin{description}
\sphinxlineitem{decoded\_symbols: str}
\sphinxAtStartPar
returns the decoded symbols

\end{description}

\end{description}

\end{fulllineitems}


\end{fulllineitems}


\sphinxstepscope


\section{Symmetric Numeral}
\label{\detokenize{symmetric_numeral:module-symmetric_numeral}}\label{\detokenize{symmetric_numeral:symmetric-numeral}}\label{\detokenize{symmetric_numeral::doc}}\index{module@\spxentry{module}!symmetric\_numeral@\spxentry{symmetric\_numeral}}\index{symmetric\_numeral@\spxentry{symmetric\_numeral}!module@\spxentry{module}}\index{SymmetricNumeral (class in symmetric\_numeral)@\spxentry{SymmetricNumeral}\spxextra{class in symmetric\_numeral}}

\begin{fulllineitems}
\phantomsection\label{\detokenize{symmetric_numeral:symmetric_numeral.SymmetricNumeral}}
\pysigstartsignatures
\pysiglinewithargsret{\sphinxbfcode{\sphinxupquote{class\DUrole{w}{  }}}\sphinxcode{\sphinxupquote{symmetric\_numeral.}}\sphinxbfcode{\sphinxupquote{SymmetricNumeral}}}{\emph{\DUrole{n}{base}\DUrole{p}{:}\DUrole{w}{  }\DUrole{n}{int}}}{}
\pysigstopsignatures
\sphinxAtStartPar
Bases: \sphinxcode{\sphinxupquote{object}}

\sphinxAtStartPar
Class for symmetric numeral encoding.
\begin{description}
\sphinxlineitem{Attributes: }\begin{description}
\sphinxlineitem{base: int}
\sphinxAtStartPar
base of numeral system

\end{description}

\end{description}
\index{decode() (symmetric\_numeral.SymmetricNumeral method)@\spxentry{decode()}\spxextra{symmetric\_numeral.SymmetricNumeral method}}

\begin{fulllineitems}
\phantomsection\label{\detokenize{symmetric_numeral:symmetric_numeral.SymmetricNumeral.decode}}
\pysigstartsignatures
\pysiglinewithargsret{\sphinxbfcode{\sphinxupquote{decode}}}{\emph{\DUrole{n}{encoded\_value}}}{}
\pysigstopsignatures
\sphinxAtStartPar
Decodes the encoded value as per as per the base
\begin{description}
\sphinxlineitem{Parameters:}\begin{description}
\sphinxlineitem{encoded\_value: base\_b}
\sphinxAtStartPar
list of digits of base b

\end{description}

\sphinxlineitem{Returns: }\begin{description}
\sphinxlineitem{decoded\_symbols: base\_b}
\sphinxAtStartPar
returns the decoded symbols

\end{description}

\end{description}

\begin{sphinxVerbatim}[commandchars=\\\{\}]
\PYG{g+gp}{\PYGZgt{}\PYGZgt{}\PYGZgt{} }\PYG{n}{s} \PYG{o}{=} \PYG{n}{SymmetricNumeral}\PYG{p}{(}\PYG{l+m+mi}{7}\PYG{p}{)}
\PYG{g+gp}{\PYGZgt{}\PYGZgt{}\PYGZgt{} }\PYG{n}{s}\PYG{o}{.}\PYG{n}{decode}\PYG{p}{(}\PYG{l+m+mi}{197833}\PYG{p}{)}
\PYG{g+go}{[1, 4, 5, 2, 5, 2, 6]}
\end{sphinxVerbatim}

\end{fulllineitems}

\index{encode() (symmetric\_numeral.SymmetricNumeral method)@\spxentry{encode()}\spxextra{symmetric\_numeral.SymmetricNumeral method}}

\begin{fulllineitems}
\phantomsection\label{\detokenize{symmetric_numeral:symmetric_numeral.SymmetricNumeral.encode}}
\pysigstartsignatures
\pysiglinewithargsret{\sphinxbfcode{\sphinxupquote{encode}}}{\emph{\DUrole{n}{message}\DUrole{p}{:}\DUrole{w}{  }\DUrole{n}{list}}}{}
\pysigstopsignatures
\sphinxAtStartPar
Encodes a set of symbols as per the base
\begin{description}
\sphinxlineitem{Parameters: }\begin{description}
\sphinxlineitem{message: list}
\sphinxAtStartPar
list of digits of base b

\end{description}

\sphinxlineitem{Returns: }\begin{description}
\sphinxlineitem{encoded\_value: base\_b}
\sphinxAtStartPar
encoding of message in base b

\end{description}

\end{description}

\begin{sphinxVerbatim}[commandchars=\\\{\}]
\PYG{g+gp}{\PYGZgt{}\PYGZgt{}\PYGZgt{} }\PYG{n}{s} \PYG{o}{=} \PYG{n}{SymmetricNumeral}\PYG{p}{(}\PYG{l+m+mi}{7}\PYG{p}{)}
\PYG{g+gp}{\PYGZgt{}\PYGZgt{}\PYGZgt{} }\PYG{n}{s}\PYG{o}{.}\PYG{n}{encode}\PYG{p}{(}\PYG{p}{[}\PYG{l+m+mi}{1}\PYG{p}{,} \PYG{l+m+mi}{4}\PYG{p}{,} \PYG{l+m+mi}{5}\PYG{p}{,} \PYG{l+m+mi}{2}\PYG{p}{,} \PYG{l+m+mi}{5}\PYG{p}{,} \PYG{l+m+mi}{2}\PYG{p}{,} \PYG{l+m+mi}{6}\PYG{p}{]}\PYG{p}{)}
\PYG{g+go}{197833}
\end{sphinxVerbatim}

\end{fulllineitems}

\index{shannon\_entropy() (symmetric\_numeral.SymmetricNumeral method)@\spxentry{shannon\_entropy()}\spxextra{symmetric\_numeral.SymmetricNumeral method}}

\begin{fulllineitems}
\phantomsection\label{\detokenize{symmetric_numeral:symmetric_numeral.SymmetricNumeral.shannon_entropy}}
\pysigstartsignatures
\pysiglinewithargsret{\sphinxbfcode{\sphinxupquote{shannon\_entropy}}}{}{}
\pysigstopsignatures
\sphinxAtStartPar
Computes the shannon’s entorpy for the base
\begin{description}
\sphinxlineitem{Retuns: float}
\sphinxAtStartPar
entropy

\end{description}

\begin{sphinxVerbatim}[commandchars=\\\{\}]
\PYG{g+gp}{\PYGZgt{}\PYGZgt{}\PYGZgt{} }\PYG{n}{s} \PYG{o}{=} \PYG{n}{SymmetricNumeral}\PYG{p}{(}\PYG{l+m+mi}{7}\PYG{p}{)}
\PYG{g+gp}{\PYGZgt{}\PYGZgt{}\PYGZgt{} }\PYG{n}{s}\PYG{o}{.}\PYG{n}{shannon\PYGZus{}entropy}\PYG{p}{(}\PYG{p}{)}
\PYG{g+go}{2.807354922057604}
\end{sphinxVerbatim}

\end{fulllineitems}


\end{fulllineitems}


\sphinxstepscope


\section{ANS}
\label{\detokenize{ANS:module-ANS}}\label{\detokenize{ANS:ans}}\label{\detokenize{ANS::doc}}\index{module@\spxentry{module}!ANS@\spxentry{ANS}}\index{ANS@\spxentry{ANS}!module@\spxentry{module}}\index{rANS (class in ANS)@\spxentry{rANS}\spxextra{class in ANS}}

\begin{fulllineitems}
\phantomsection\label{\detokenize{ANS:ANS.rANS}}
\pysigstartsignatures
\pysiglinewithargsret{\sphinxbfcode{\sphinxupquote{class\DUrole{w}{  }}}\sphinxcode{\sphinxupquote{ANS.}}\sphinxbfcode{\sphinxupquote{rANS}}}{\emph{\DUrole{n}{symbols}\DUrole{p}{:}\DUrole{w}{  }\DUrole{n}{list}}, \emph{\DUrole{n}{frequency}\DUrole{p}{:}\DUrole{w}{  }\DUrole{n}{list}}}{}
\pysigstopsignatures
\sphinxAtStartPar
Bases: {\hyperref[\detokenize{core:core.data.Data}]{\sphinxcrossref{\sphinxcode{\sphinxupquote{Data}}}}}

\sphinxAtStartPar
rANS compressor and decompressor class. Inherits the data class.
\begin{description}
\sphinxlineitem{Attributes:}\begin{description}
\sphinxlineitem{symbols: list}
\sphinxAtStartPar
list of all possible symbols

\sphinxlineitem{frequency: list}
\sphinxAtStartPar
list of symbol frequency

\end{description}

\end{description}
\index{decode() (ANS.rANS method)@\spxentry{decode()}\spxextra{ANS.rANS method}}

\begin{fulllineitems}
\phantomsection\label{\detokenize{ANS:ANS.rANS.decode}}
\pysigstartsignatures
\pysiglinewithargsret{\sphinxbfcode{\sphinxupquote{decode}}}{\emph{\DUrole{n}{encoded\_value}\DUrole{p}{:}\DUrole{w}{  }\DUrole{n}{str}}, \emph{\DUrole{n}{msg\_len}}}{{ $\rightarrow$ list}}
\pysigstopsignatures
\sphinxAtStartPar
rANS decode function
\begin{description}
\sphinxlineitem{Parameters: }\begin{description}
\sphinxlineitem{encoded\_value: int}
\sphinxAtStartPar
final state after encoding 
this function inherits the probability distribuiton of the symbols.
This function assumes that the probability distribuiton is know and the class is instantiated

\end{description}

\sphinxlineitem{Returns:}\begin{description}
\sphinxlineitem{symbols: list}
\sphinxAtStartPar
the decoded symbols in reverse order

\end{description}

\end{description}

\begin{sphinxVerbatim}[commandchars=\\\{\}]
\PYG{g+gp}{\PYGZgt{}\PYGZgt{}\PYGZgt{} }\PYG{n}{symbols} \PYG{o}{=} \PYG{p}{[}\PYG{l+s+s1}{\PYGZsq{}}\PYG{l+s+s1}{a}\PYG{l+s+s1}{\PYGZsq{}}\PYG{p}{,} \PYG{l+s+s1}{\PYGZsq{}}\PYG{l+s+s1}{b}\PYG{l+s+s1}{\PYGZsq{}}\PYG{p}{,} \PYG{l+s+s1}{\PYGZsq{}}\PYG{l+s+s1}{c}\PYG{l+s+s1}{\PYGZsq{}}\PYG{p}{]}
\PYG{g+gp}{\PYGZgt{}\PYGZgt{}\PYGZgt{} }\PYG{n}{freq} \PYG{o}{=} \PYG{p}{[}\PYG{l+m+mi}{5}\PYG{p}{,} \PYG{l+m+mi}{5}\PYG{p}{,} \PYG{l+m+mi}{2}\PYG{p}{]}
\PYG{g+gp}{\PYGZgt{}\PYGZgt{}\PYGZgt{} }\PYG{n}{a} \PYG{o}{=} \PYG{n}{rANS}\PYG{p}{(}\PYG{n}{symbols}\PYG{p}{,} \PYG{n}{freq}\PYG{p}{)}
\PYG{g+gp}{\PYGZgt{}\PYGZgt{}\PYGZgt{} }\PYG{n}{a}\PYG{o}{.}\PYG{n}{decode}\PYG{p}{(}\PYG{l+m+mi}{1242}\PYG{p}{,}\PYG{l+m+mi}{6}\PYG{p}{)}
\PYG{g+go}{[\PYGZsq{}a\PYGZsq{}, \PYGZsq{}b\PYGZsq{}, \PYGZsq{}c\PYGZsq{}, \PYGZsq{}c\PYGZsq{}, \PYGZsq{}a\PYGZsq{}, \PYGZsq{}b\PYGZsq{}]}
\end{sphinxVerbatim}

\end{fulllineitems}

\index{encode() (ANS.rANS method)@\spxentry{encode()}\spxextra{ANS.rANS method}}

\begin{fulllineitems}
\phantomsection\label{\detokenize{ANS:ANS.rANS.encode}}
\pysigstartsignatures
\pysiglinewithargsret{\sphinxbfcode{\sphinxupquote{encode}}}{\emph{\DUrole{n}{msg}\DUrole{p}{:}\DUrole{w}{  }\DUrole{n}{list}}, \emph{\DUrole{n}{start\_state}\DUrole{p}{:}\DUrole{w}{  }\DUrole{n}{int}}}{{ $\rightarrow$ Tuple\DUrole{p}{{[}}str\DUrole{p}{,}\DUrole{w}{  }int\DUrole{p}{{]}}}}
\pysigstopsignatures
\sphinxAtStartPar
rANS encode function
\begin{description}
\sphinxlineitem{Parameters:}\begin{description}
\sphinxlineitem{data: list}
\sphinxAtStartPar
data to be encoded. Has to be a list

\end{description}

\sphinxlineitem{Returns:}\begin{description}
\sphinxlineitem{final\_state: int }
\sphinxAtStartPar
final encoded value

\end{description}

\end{description}

\begin{sphinxVerbatim}[commandchars=\\\{\}]
\PYG{g+gp}{\PYGZgt{}\PYGZgt{}\PYGZgt{} }\PYG{n}{symbols} \PYG{o}{=} \PYG{p}{[}\PYG{l+s+s1}{\PYGZsq{}}\PYG{l+s+s1}{a}\PYG{l+s+s1}{\PYGZsq{}}\PYG{p}{,} \PYG{l+s+s1}{\PYGZsq{}}\PYG{l+s+s1}{b}\PYG{l+s+s1}{\PYGZsq{}}\PYG{p}{,} \PYG{l+s+s1}{\PYGZsq{}}\PYG{l+s+s1}{c}\PYG{l+s+s1}{\PYGZsq{}}\PYG{p}{]}
\PYG{g+gp}{\PYGZgt{}\PYGZgt{}\PYGZgt{} }\PYG{n}{freq} \PYG{o}{=} \PYG{p}{[}\PYG{l+m+mi}{5}\PYG{p}{,} \PYG{l+m+mi}{5}\PYG{p}{,} \PYG{l+m+mi}{2}\PYG{p}{]}
\PYG{g+gp}{\PYGZgt{}\PYGZgt{}\PYGZgt{} }\PYG{n}{a} \PYG{o}{=} \PYG{n}{rANS}\PYG{p}{(}\PYG{n}{symbols}\PYG{p}{,} \PYG{n}{freq}\PYG{p}{)}
\PYG{g+gp}{\PYGZgt{}\PYGZgt{}\PYGZgt{} }\PYG{n}{a}\PYG{o}{.}\PYG{n}{encode}\PYG{p}{(}\PYG{p}{[}\PYG{l+s+s1}{\PYGZsq{}}\PYG{l+s+s1}{a}\PYG{l+s+s1}{\PYGZsq{}}\PYG{p}{,} \PYG{l+s+s1}{\PYGZsq{}}\PYG{l+s+s1}{b}\PYG{l+s+s1}{\PYGZsq{}}\PYG{p}{,} \PYG{l+s+s1}{\PYGZsq{}}\PYG{l+s+s1}{c}\PYG{l+s+s1}{\PYGZsq{}}\PYG{p}{,} \PYG{l+s+s1}{\PYGZsq{}}\PYG{l+s+s1}{c}\PYG{l+s+s1}{\PYGZsq{}}\PYG{p}{,} \PYG{l+s+s1}{\PYGZsq{}}\PYG{l+s+s1}{a}\PYG{l+s+s1}{\PYGZsq{}}\PYG{p}{,} \PYG{l+s+s1}{\PYGZsq{}}\PYG{l+s+s1}{b}\PYG{l+s+s1}{\PYGZsq{}}\PYG{p}{]}\PYG{p}{,} \PYG{l+m+mi}{0}\PYG{p}{)}
\PYG{g+go}{1242}
\end{sphinxVerbatim}

\end{fulllineitems}

\index{rANS\_decode\_step() (ANS.rANS method)@\spxentry{rANS\_decode\_step()}\spxextra{ANS.rANS method}}

\begin{fulllineitems}
\phantomsection\label{\detokenize{ANS:ANS.rANS.rANS_decode_step}}
\pysigstartsignatures
\pysiglinewithargsret{\sphinxbfcode{\sphinxupquote{rANS\_decode\_step}}}{\emph{\DUrole{n}{x\_next}\DUrole{p}{:}\DUrole{w}{  }\DUrole{n}{int}}}{{ $\rightarrow$ tuple}}
\pysigstopsignatures
\sphinxAtStartPar
Decoding step function

\end{fulllineitems}

\index{rANS\_encode\_step() (ANS.rANS method)@\spxentry{rANS\_encode\_step()}\spxextra{ANS.rANS method}}

\begin{fulllineitems}
\phantomsection\label{\detokenize{ANS:ANS.rANS.rANS_encode_step}}
\pysigstartsignatures
\pysiglinewithargsret{\sphinxbfcode{\sphinxupquote{rANS\_encode\_step}}}{\emph{\DUrole{n}{symbol}}, \emph{\DUrole{n}{x\_prev}\DUrole{p}{:}\DUrole{w}{  }\DUrole{n}{int}}}{{ $\rightarrow$ int}}
\pysigstopsignatures
\sphinxAtStartPar
Encoding step function

\end{fulllineitems}

\index{rANS\_encoding\_table() (ANS.rANS method)@\spxentry{rANS\_encoding\_table()}\spxextra{ANS.rANS method}}

\begin{fulllineitems}
\phantomsection\label{\detokenize{ANS:ANS.rANS.rANS_encoding_table}}
\pysigstartsignatures
\pysiglinewithargsret{\sphinxbfcode{\sphinxupquote{rANS\_encoding\_table}}}{\emph{\DUrole{n}{final\_state}}}{{ $\rightarrow$ pandas.DataFrame}}
\pysigstopsignatures
\sphinxAtStartPar
Returns the rANS encoding table. The format is similar to Dudak’s paper. 
Parameters:
\begin{quote}

\sphinxAtStartPar
final\_state: the final state table should contain
\end{quote}
\begin{description}
\sphinxlineitem{Returns: }\begin{description}
\sphinxlineitem{table: pd.Dataframe}
\sphinxAtStartPar
returns and pandas dataframe and prints the dataframe.

\end{description}

\end{description}

\end{fulllineitems}


\end{fulllineitems}

\index{rANSDecoder (class in ANS)@\spxentry{rANSDecoder}\spxextra{class in ANS}}

\begin{fulllineitems}
\phantomsection\label{\detokenize{ANS:ANS.rANSDecoder}}
\pysigstartsignatures
\pysiglinewithargsret{\sphinxbfcode{\sphinxupquote{class\DUrole{w}{  }}}\sphinxcode{\sphinxupquote{ANS.}}\sphinxbfcode{\sphinxupquote{rANSDecoder}}}{\emph{\DUrole{n}{symbols}\DUrole{p}{:}\DUrole{w}{  }\DUrole{n}{list}}, \emph{\DUrole{n}{frequency}\DUrole{p}{:}\DUrole{w}{  }\DUrole{n}{list}}}{}
\pysigstopsignatures
\sphinxAtStartPar
Bases: {\hyperref[\detokenize{core:core.data.Data}]{\sphinxcrossref{\sphinxcode{\sphinxupquote{Data}}}}}

\sphinxAtStartPar
rANSDecoder class for decoding given symbols and frequency.
\begin{description}
\sphinxlineitem{Parmaeters:}\begin{description}
\sphinxlineitem{symbols: list}
\sphinxAtStartPar
a list of symbols

\sphinxlineitem{frequency: list}
\sphinxAtStartPar
frequency distribuiton list

\end{description}

\end{description}
\index{decode() (ANS.rANSDecoder method)@\spxentry{decode()}\spxextra{ANS.rANSDecoder method}}

\begin{fulllineitems}
\phantomsection\label{\detokenize{ANS:ANS.rANSDecoder.decode}}
\pysigstartsignatures
\pysiglinewithargsret{\sphinxbfcode{\sphinxupquote{decode}}}{\emph{\DUrole{n}{encoded\_value}\DUrole{p}{:}\DUrole{w}{  }\DUrole{n}{str}}, \emph{\DUrole{n}{msg\_len}\DUrole{p}{:}\DUrole{w}{  }\DUrole{n}{int}}}{}
\pysigstopsignatures
\sphinxAtStartPar
Function to decode, give the correct order
Parameters:
\begin{quote}
\begin{description}
\sphinxlineitem{encoded\_value: int}
\sphinxAtStartPar
final state after encoding

\end{description}
\end{quote}
\begin{description}
\sphinxlineitem{Returns:}\begin{description}
\sphinxlineitem{decoded symbols: list}
\sphinxAtStartPar
list of decoded symbols

\end{description}

\end{description}

\begin{sphinxVerbatim}[commandchars=\\\{\}]
\PYG{g+gp}{\PYGZgt{}\PYGZgt{}\PYGZgt{} }\PYG{n}{symbols} \PYG{o}{=} \PYG{p}{[}\PYG{l+s+s1}{\PYGZsq{}}\PYG{l+s+s1}{a}\PYG{l+s+s1}{\PYGZsq{}}\PYG{p}{,} \PYG{l+s+s1}{\PYGZsq{}}\PYG{l+s+s1}{b}\PYG{l+s+s1}{\PYGZsq{}}\PYG{p}{,} \PYG{l+s+s1}{\PYGZsq{}}\PYG{l+s+s1}{c}\PYG{l+s+s1}{\PYGZsq{}}\PYG{p}{]}
\PYG{g+gp}{\PYGZgt{}\PYGZgt{}\PYGZgt{} }\PYG{n}{freq} \PYG{o}{=} \PYG{p}{[}\PYG{l+m+mi}{5}\PYG{p}{,} \PYG{l+m+mi}{5}\PYG{p}{,} \PYG{l+m+mi}{2}\PYG{p}{]}
\PYG{g+gp}{\PYGZgt{}\PYGZgt{}\PYGZgt{} }\PYG{n}{a} \PYG{o}{=} \PYG{n}{rANSDecoder}\PYG{p}{(}\PYG{n}{symbols}\PYG{p}{,} \PYG{n}{freq}\PYG{p}{)}
\PYG{g+gp}{\PYGZgt{}\PYGZgt{}\PYGZgt{} }\PYG{n}{a}\PYG{o}{.}\PYG{n}{decode}\PYG{p}{(}\PYG{l+m+mi}{1242}\PYG{p}{,}\PYG{l+m+mi}{6}\PYG{p}{)}
\PYG{g+go}{[\PYGZsq{}a\PYGZsq{}, \PYGZsq{}b\PYGZsq{}, \PYGZsq{}c\PYGZsq{}, \PYGZsq{}c\PYGZsq{}, \PYGZsq{}a\PYGZsq{}, \PYGZsq{}b\PYGZsq{}]}
\end{sphinxVerbatim}

\end{fulllineitems}


\end{fulllineitems}


\sphinxstepscope


\section{uABS}
\label{\detokenize{uABS:module-uABS}}\label{\detokenize{uABS:uabs}}\label{\detokenize{uABS::doc}}\index{module@\spxentry{module}!uABS@\spxentry{uABS}}\index{uABS@\spxentry{uABS}!module@\spxentry{module}}\index{uABS (class in uABS)@\spxentry{uABS}\spxextra{class in uABS}}

\begin{fulllineitems}
\phantomsection\label{\detokenize{uABS:uABS.uABS}}
\pysigstartsignatures
\pysiglinewithargsret{\sphinxbfcode{\sphinxupquote{class\DUrole{w}{  }}}\sphinxcode{\sphinxupquote{uABS.}}\sphinxbfcode{\sphinxupquote{uABS}}}{\emph{\DUrole{n}{p}}}{}
\pysigstopsignatures
\sphinxAtStartPar
Bases: \sphinxcode{\sphinxupquote{object}}
\index{decode() (uABS.uABS method)@\spxentry{decode()}\spxextra{uABS.uABS method}}

\begin{fulllineitems}
\phantomsection\label{\detokenize{uABS:uABS.uABS.decode}}
\pysigstartsignatures
\pysiglinewithargsret{\sphinxbfcode{\sphinxupquote{decode}}}{\emph{\DUrole{n}{final\_state}}, \emph{\DUrole{n}{msg\_len}\DUrole{p}{:}\DUrole{w}{  }\DUrole{n}{int}}}{}
\pysigstopsignatures
\end{fulllineitems}

\index{encode() (uABS.uABS method)@\spxentry{encode()}\spxextra{uABS.uABS method}}

\begin{fulllineitems}
\phantomsection\label{\detokenize{uABS:uABS.uABS.encode}}
\pysigstartsignatures
\pysiglinewithargsret{\sphinxbfcode{\sphinxupquote{encode}}}{\emph{\DUrole{n}{msg}\DUrole{p}{:}\DUrole{w}{  }\DUrole{n}{str}}, \emph{\DUrole{n}{initial\_state}\DUrole{o}{=}\DUrole{default_value}{0}}}{}
\pysigstopsignatures
\end{fulllineitems}

\index{shannon\_entropy() (uABS.uABS method)@\spxentry{shannon\_entropy()}\spxextra{uABS.uABS method}}

\begin{fulllineitems}
\phantomsection\label{\detokenize{uABS:uABS.uABS.shannon_entropy}}
\pysigstartsignatures
\pysiglinewithargsret{\sphinxbfcode{\sphinxupquote{shannon\_entropy}}}{}{}
\pysigstopsignatures
\end{fulllineitems}

\index{uABS\_decode\_step() (uABS.uABS method)@\spxentry{uABS\_decode\_step()}\spxextra{uABS.uABS method}}

\begin{fulllineitems}
\phantomsection\label{\detokenize{uABS:uABS.uABS.uABS_decode_step}}
\pysigstartsignatures
\pysiglinewithargsret{\sphinxbfcode{\sphinxupquote{uABS\_decode\_step}}}{\emph{\DUrole{n}{x}}}{}
\pysigstopsignatures
\end{fulllineitems}

\index{uABS\_encode\_step() (uABS.uABS method)@\spxentry{uABS\_encode\_step()}\spxextra{uABS.uABS method}}

\begin{fulllineitems}
\phantomsection\label{\detokenize{uABS:uABS.uABS.uABS_encode_step}}
\pysigstartsignatures
\pysiglinewithargsret{\sphinxbfcode{\sphinxupquote{uABS\_encode\_step}}}{\emph{\DUrole{n}{s}\DUrole{p}{:}\DUrole{w}{  }\DUrole{n}{str}}, \emph{\DUrole{n}{x\_prev}\DUrole{p}{:}\DUrole{w}{  }\DUrole{n}{int}}}{}
\pysigstopsignatures
\end{fulllineitems}


\end{fulllineitems}


\sphinxstepscope


\section{Streaming ANS}
\label{\detokenize{sANS:module-sANS}}\label{\detokenize{sANS:streaming-ans}}\label{\detokenize{sANS::doc}}\index{module@\spxentry{module}!sANS@\spxentry{sANS}}\index{sANS@\spxentry{sANS}!module@\spxentry{module}}\index{sANS (class in sANS)@\spxentry{sANS}\spxextra{class in sANS}}

\begin{fulllineitems}
\phantomsection\label{\detokenize{sANS:sANS.sANS}}
\pysigstartsignatures
\pysiglinewithargsret{\sphinxbfcode{\sphinxupquote{class\DUrole{w}{  }}}\sphinxcode{\sphinxupquote{sANS.}}\sphinxbfcode{\sphinxupquote{sANS}}}{\emph{\DUrole{n}{symbols}\DUrole{p}{:}\DUrole{w}{  }\DUrole{n}{list}}, \emph{\DUrole{n}{frequency}\DUrole{p}{:}\DUrole{w}{  }\DUrole{n}{list}}}{}
\pysigstopsignatures
\sphinxAtStartPar
Bases: {\hyperref[\detokenize{core:core.data.Data}]{\sphinxcrossref{\sphinxcode{\sphinxupquote{Data}}}}}

\sphinxAtStartPar
sANS compressor and decompressor class. Inherits the data class.
Initializes the rANS class. 
Attributes:
\begin{quote}
\begin{description}
\sphinxlineitem{symbols: list}
\sphinxAtStartPar
list of all possible symbols

\sphinxlineitem{frequency: list}
\sphinxAtStartPar
list of symbol frequency

\end{description}
\end{quote}
\index{decode() (sANS.sANS method)@\spxentry{decode()}\spxextra{sANS.sANS method}}

\begin{fulllineitems}
\phantomsection\label{\detokenize{sANS:sANS.sANS.decode}}
\pysigstartsignatures
\pysiglinewithargsret{\sphinxbfcode{\sphinxupquote{decode}}}{\emph{\DUrole{n}{x}\DUrole{p}{:}\DUrole{w}{  }\DUrole{n}{str}}, \emph{\DUrole{n}{bit\_array}\DUrole{p}{:}\DUrole{w}{  }\DUrole{n}{bitarray.bitarray}}}{}
\pysigstopsignatures
\sphinxAtStartPar
sANS decode function
\begin{description}
\sphinxlineitem{Parameters: }\begin{description}
\sphinxlineitem{encoded\_value: int}
\sphinxAtStartPar
final state after encoding 
this function inherits the probability distribuiton of the symbols.
This function assumes that the probability distribuiton is know and the class is instantiated

\sphinxlineitem{bit\_array: bitarray.bitarray}
\sphinxAtStartPar
the bit output from renormalization

\end{description}

\sphinxlineitem{Returns:}\begin{description}
\sphinxlineitem{symbols: list}
\sphinxAtStartPar
the decoded symbols in reverse order

\end{description}

\end{description}

\begin{sphinxVerbatim}[commandchars=\\\{\}]
\PYG{g+gp}{\PYGZgt{}\PYGZgt{}\PYGZgt{} }\PYG{n}{symbols} \PYG{o}{=} \PYG{p}{[}\PYG{l+s+s1}{\PYGZsq{}}\PYG{l+s+s1}{a}\PYG{l+s+s1}{\PYGZsq{}}\PYG{p}{,} \PYG{l+s+s1}{\PYGZsq{}}\PYG{l+s+s1}{b}\PYG{l+s+s1}{\PYGZsq{}}\PYG{p}{,} \PYG{l+s+s1}{\PYGZsq{}}\PYG{l+s+s1}{c}\PYG{l+s+s1}{\PYGZsq{}}\PYG{p}{]}
\PYG{g+gp}{\PYGZgt{}\PYGZgt{}\PYGZgt{} }\PYG{n}{freq} \PYG{o}{=} \PYG{p}{[}\PYG{l+m+mi}{5}\PYG{p}{,} \PYG{l+m+mi}{5}\PYG{p}{,} \PYG{l+m+mi}{2}\PYG{p}{]}
\PYG{g+gp}{\PYGZgt{}\PYGZgt{}\PYGZgt{} }\PYG{n}{a} \PYG{o}{=} \PYG{n}{rANS}\PYG{p}{(}\PYG{n}{symbols}\PYG{p}{,} \PYG{n}{freq}\PYG{p}{)}
\PYG{g+gp}{\PYGZgt{}\PYGZgt{}\PYGZgt{} }\PYG{n}{a}\PYG{o}{.}\PYG{n}{decode}\PYG{p}{(}\PYG{l+m+mi}{18}\PYG{p}{,} \PYG{n}{bitarray}\PYG{o}{.}\PYG{n}{bitarray}\PYG{p}{(}\PYG{l+s+s1}{\PYGZsq{}}\PYG{l+s+s1}{00110011010}\PYG{l+s+s1}{\PYGZsq{}}\PYG{p}{)}\PYG{p}{)}
\PYG{g+go}{[\PYGZsq{}a\PYGZsq{}, \PYGZsq{}b\PYGZsq{}, \PYGZsq{}c\PYGZsq{}, \PYGZsq{}c\PYGZsq{}, \PYGZsq{}a\PYGZsq{}, \PYGZsq{}b\PYGZsq{}]}
\end{sphinxVerbatim}

\end{fulllineitems}

\index{encode() (sANS.sANS method)@\spxentry{encode()}\spxextra{sANS.sANS method}}

\begin{fulllineitems}
\phantomsection\label{\detokenize{sANS:sANS.sANS.encode}}
\pysigstartsignatures
\pysiglinewithargsret{\sphinxbfcode{\sphinxupquote{encode}}}{\emph{\DUrole{n}{msg}\DUrole{p}{:}\DUrole{w}{  }\DUrole{n}{list}}}{}
\pysigstopsignatures
\sphinxAtStartPar
sANS encode function
\begin{description}
\sphinxlineitem{Parameters:}\begin{description}
\sphinxlineitem{msg: list}
\sphinxAtStartPar
data to be encoded. Has to be a list

\sphinxlineitem{initial\_state: int}
\sphinxAtStartPar
initial state must be \textgreater{}= sum of freq

\end{description}

\sphinxlineitem{Returns:}\begin{description}
\sphinxlineitem{final\_state: int }
\sphinxAtStartPar
final encoded value

\sphinxlineitem{bit\_output: bitarray.bitarray}
\sphinxAtStartPar
bit output from rescaling

\end{description}

\end{description}

\begin{sphinxVerbatim}[commandchars=\\\{\}]
\PYG{g+gp}{\PYGZgt{}\PYGZgt{}\PYGZgt{} }\PYG{n}{symbols} \PYG{o}{=} \PYG{p}{[}\PYG{l+s+s1}{\PYGZsq{}}\PYG{l+s+s1}{a}\PYG{l+s+s1}{\PYGZsq{}}\PYG{p}{,} \PYG{l+s+s1}{\PYGZsq{}}\PYG{l+s+s1}{b}\PYG{l+s+s1}{\PYGZsq{}}\PYG{p}{,} \PYG{l+s+s1}{\PYGZsq{}}\PYG{l+s+s1}{c}\PYG{l+s+s1}{\PYGZsq{}}\PYG{p}{]}
\PYG{g+gp}{\PYGZgt{}\PYGZgt{}\PYGZgt{} }\PYG{n}{freq} \PYG{o}{=} \PYG{p}{[}\PYG{l+m+mi}{5}\PYG{p}{,} \PYG{l+m+mi}{5}\PYG{p}{,} \PYG{l+m+mi}{2}\PYG{p}{]}
\PYG{g+gp}{\PYGZgt{}\PYGZgt{}\PYGZgt{} }\PYG{n}{a} \PYG{o}{=} \PYG{n}{sANS}\PYG{p}{(}\PYG{n}{symbols}\PYG{p}{,} \PYG{n}{freq}\PYG{p}{)}
\PYG{g+gp}{\PYGZgt{}\PYGZgt{}\PYGZgt{} }\PYG{n}{a}\PYG{o}{.}\PYG{n}{encode}\PYG{p}{(}\PYG{p}{[}\PYG{l+s+s1}{\PYGZsq{}}\PYG{l+s+s1}{a}\PYG{l+s+s1}{\PYGZsq{}}\PYG{p}{,} \PYG{l+s+s1}{\PYGZsq{}}\PYG{l+s+s1}{b}\PYG{l+s+s1}{\PYGZsq{}}\PYG{p}{,} \PYG{l+s+s1}{\PYGZsq{}}\PYG{l+s+s1}{c}\PYG{l+s+s1}{\PYGZsq{}}\PYG{p}{,} \PYG{l+s+s1}{\PYGZsq{}}\PYG{l+s+s1}{c}\PYG{l+s+s1}{\PYGZsq{}}\PYG{p}{,} \PYG{l+s+s1}{\PYGZsq{}}\PYG{l+s+s1}{a}\PYG{l+s+s1}{\PYGZsq{}}\PYG{p}{,} \PYG{l+s+s1}{\PYGZsq{}}\PYG{l+s+s1}{b}\PYG{l+s+s1}{\PYGZsq{}}\PYG{p}{]}\PYG{p}{,} \PYG{l+m+mi}{14}\PYG{p}{)}
\PYG{g+go}{18 bitarray(\PYGZsq{}00110011010\PYGZsq{})}
\end{sphinxVerbatim}

\end{fulllineitems}


\end{fulllineitems}

\index{sANSDecoder (class in sANS)@\spxentry{sANSDecoder}\spxextra{class in sANS}}

\begin{fulllineitems}
\phantomsection\label{\detokenize{sANS:sANS.sANSDecoder}}
\pysigstartsignatures
\pysiglinewithargsret{\sphinxbfcode{\sphinxupquote{class\DUrole{w}{  }}}\sphinxcode{\sphinxupquote{sANS.}}\sphinxbfcode{\sphinxupquote{sANSDecoder}}}{\emph{\DUrole{n}{symbols}\DUrole{p}{:}\DUrole{w}{  }\DUrole{n}{list}}, \emph{\DUrole{n}{frequency}\DUrole{p}{:}\DUrole{w}{  }\DUrole{n}{list}}}{}
\pysigstopsignatures
\sphinxAtStartPar
Bases: {\hyperref[\detokenize{core:core.data.Data}]{\sphinxcrossref{\sphinxcode{\sphinxupquote{Data}}}}}

\sphinxAtStartPar
rANSDecoder class for decoding given symbols and frequency.
initializes the sANS class. 
Parmaeters:
\begin{quote}
\begin{description}
\sphinxlineitem{symbols: list}
\sphinxAtStartPar
a list of symbols

\sphinxlineitem{frequency: list}
\sphinxAtStartPar
frequency distribuiton list

\end{description}
\end{quote}
\index{decode() (sANS.sANSDecoder method)@\spxentry{decode()}\spxextra{sANS.sANSDecoder method}}

\begin{fulllineitems}
\phantomsection\label{\detokenize{sANS:sANS.sANSDecoder.decode}}
\pysigstartsignatures
\pysiglinewithargsret{\sphinxbfcode{\sphinxupquote{decode}}}{\emph{\DUrole{n}{x}\DUrole{p}{:}\DUrole{w}{  }\DUrole{n}{str}}, \emph{\DUrole{n}{bit\_array}\DUrole{p}{:}\DUrole{w}{  }\DUrole{n}{bitarray.bitarray}}}{}
\pysigstopsignatures
\sphinxAtStartPar
Function to decode, give the correct order
Parameters:
\begin{quote}
\begin{description}
\sphinxlineitem{encoded\_value: int}
\sphinxAtStartPar
final state after encoding

\end{description}
\end{quote}
\begin{description}
\sphinxlineitem{Returns:}\begin{description}
\sphinxlineitem{decoded symbols: list}
\sphinxAtStartPar
list of decoded symbols

\end{description}

\end{description}

\begin{sphinxVerbatim}[commandchars=\\\{\}]
\PYG{g+gp}{\PYGZgt{}\PYGZgt{}\PYGZgt{} }\PYG{n}{symbols} \PYG{o}{=} \PYG{p}{[}\PYG{l+s+s1}{\PYGZsq{}}\PYG{l+s+s1}{a}\PYG{l+s+s1}{\PYGZsq{}}\PYG{p}{,} \PYG{l+s+s1}{\PYGZsq{}}\PYG{l+s+s1}{b}\PYG{l+s+s1}{\PYGZsq{}}\PYG{p}{,} \PYG{l+s+s1}{\PYGZsq{}}\PYG{l+s+s1}{c}\PYG{l+s+s1}{\PYGZsq{}}\PYG{p}{]}
\PYG{g+gp}{\PYGZgt{}\PYGZgt{}\PYGZgt{} }\PYG{n}{freq} \PYG{o}{=} \PYG{p}{[}\PYG{l+m+mi}{5}\PYG{p}{,} \PYG{l+m+mi}{5}\PYG{p}{,} \PYG{l+m+mi}{2}\PYG{p}{]}
\PYG{g+gp}{\PYGZgt{}\PYGZgt{}\PYGZgt{} }\PYG{n}{a} \PYG{o}{=} \PYG{n}{sANSDecoder}\PYG{p}{(}\PYG{n}{symbols}\PYG{p}{,} \PYG{n}{freq}\PYG{p}{)}
\PYG{g+gp}{\PYGZgt{}\PYGZgt{}\PYGZgt{} }\PYG{n}{a}\PYG{o}{.}\PYG{n}{decode}\PYG{p}{(}\PYG{l+m+mi}{18}\PYG{p}{,} \PYG{n}{bitarray}\PYG{o}{.}\PYG{n}{bitarray}\PYG{p}{(}\PYG{l+s+s1}{\PYGZsq{}}\PYG{l+s+s1}{00110011010}\PYG{l+s+s1}{\PYGZsq{}}\PYG{p}{)}\PYG{p}{)}
\PYG{g+go}{[\PYGZsq{}a\PYGZsq{}, \PYGZsq{}b\PYGZsq{}, \PYGZsq{}c\PYGZsq{}, \PYGZsq{}c\PYGZsq{}, \PYGZsq{}a\PYGZsq{}, \PYGZsq{}b\PYGZsq{}]}
\end{sphinxVerbatim}

\end{fulllineitems}


\end{fulllineitems}


\sphinxstepscope


\section{File Compressor}
\label{\detokenize{file_compressor:module-file_compressor}}\label{\detokenize{file_compressor:file-compressor}}\label{\detokenize{file_compressor::doc}}\index{module@\spxentry{module}!file\_compressor@\spxentry{file\_compressor}}\index{file\_compressor@\spxentry{file\_compressor}!module@\spxentry{module}}\index{FileCompressor (class in file\_compressor)@\spxentry{FileCompressor}\spxextra{class in file\_compressor}}

\begin{fulllineitems}
\phantomsection\label{\detokenize{file_compressor:file_compressor.FileCompressor}}
\pysigstartsignatures
\pysiglinewithargsret{\sphinxbfcode{\sphinxupquote{class\DUrole{w}{  }}}\sphinxcode{\sphinxupquote{file\_compressor.}}\sphinxbfcode{\sphinxupquote{FileCompressor}}}{\emph{\DUrole{n}{file}}}{}
\pysigstopsignatures
\sphinxAtStartPar
Bases: \sphinxcode{\sphinxupquote{object}}

\sphinxAtStartPar
Compresses a file


\subsection{parameter:}
\label{\detokenize{file_compressor:parameter}}\begin{description}
\sphinxlineitem{file: str}
\sphinxAtStartPar
file location

\sphinxlineitem{returns: }
\sphinxAtStartPar
final\_rANS state

\end{description}

\sphinxAtStartPar
Note: The decode function can be used given the class has been configured with correct prob distn.
\index{create\_aux\_file() (file\_compressor.FileCompressor method)@\spxentry{create\_aux\_file()}\spxextra{file\_compressor.FileCompressor method}}

\begin{fulllineitems}
\phantomsection\label{\detokenize{file_compressor:file_compressor.FileCompressor.create_aux_file}}
\pysigstartsignatures
\pysiglinewithargsret{\sphinxbfcode{\sphinxupquote{create\_aux\_file}}}{}{}
\pysigstopsignatures
\sphinxAtStartPar
creates an auxiliary file with symbols and their frequency distribution: for decompressor
consists of:
\begin{quote}

\sphinxAtStartPar
symbols, frequency, file\_name
\end{quote}

\sphinxAtStartPar
using this becomes counter intuituve as the aux\_file will have size \textgreater{} original file.

\end{fulllineitems}

\index{decode() (file\_compressor.FileCompressor method)@\spxentry{decode()}\spxextra{file\_compressor.FileCompressor method}}

\begin{fulllineitems}
\phantomsection\label{\detokenize{file_compressor:file_compressor.FileCompressor.decode}}
\pysigstartsignatures
\pysiglinewithargsret{\sphinxbfcode{\sphinxupquote{decode}}}{\emph{\DUrole{n}{encoded\_value}\DUrole{p}{:}\DUrole{w}{  }\DUrole{n}{int}}}{}
\pysigstopsignatures
\end{fulllineitems}

\index{file\_encode() (file\_compressor.FileCompressor method)@\spxentry{file\_encode()}\spxextra{file\_compressor.FileCompressor method}}

\begin{fulllineitems}
\phantomsection\label{\detokenize{file_compressor:file_compressor.FileCompressor.file_encode}}
\pysigstartsignatures
\pysiglinewithargsret{\sphinxbfcode{\sphinxupquote{file\_encode}}}{\emph{\DUrole{n}{compressor}}}{}
\pysigstopsignatures
\sphinxAtStartPar
Given a compression algorithm compressed a file. 
Parameters:
\begin{quote}
\begin{description}
\sphinxlineitem{compressor: str}
\sphinxAtStartPar
compression algorithm either from huffman, airhtmetic, range, rANS, sANS

\end{description}
\end{quote}
\begin{description}
\sphinxlineitem{Returns:}\begin{description}
\sphinxlineitem{encodec\_value: int | str | Tuple}
\sphinxAtStartPar
the encoded value can be anything depending on the output of the compression algorithm.

\end{description}

\end{description}

\end{fulllineitems}

\index{summary() (file\_compressor.FileCompressor method)@\spxentry{summary()}\spxextra{file\_compressor.FileCompressor method}}

\begin{fulllineitems}
\phantomsection\label{\detokenize{file_compressor:file_compressor.FileCompressor.summary}}
\pysigstartsignatures
\pysiglinewithargsret{\sphinxbfcode{\sphinxupquote{summary}}}{}{}
\pysigstopsignatures
\sphinxAtStartPar
function that gives summary of the file
:
file\_name
file\_size
file\_creation\_tiem
file\_modification\_time
total\_symbols
unique\_symbols
frequeency\_dist
shannon\_entrory
compressed\_size
compression\_ratio

\end{fulllineitems}


\end{fulllineitems}


\sphinxstepscope


\section{Utils}
\label{\detokenize{utils:utils}}\label{\detokenize{utils::doc}}

\subsection{utils.ac\_utils module}
\label{\detokenize{utils:module-utils.ac_utils}}\label{\detokenize{utils:utils-ac-utils-module}}\index{module@\spxentry{module}!utils.ac\_utils@\spxentry{utils.ac\_utils}}\index{utils.ac\_utils@\spxentry{utils.ac\_utils}!module@\spxentry{module}}\index{decToBinConversion() (in module utils.ac\_utils)@\spxentry{decToBinConversion()}\spxextra{in module utils.ac\_utils}}

\begin{fulllineitems}
\phantomsection\label{\detokenize{utils:utils.ac_utils.decToBinConversion}}
\pysigstartsignatures
\pysiglinewithargsret{\sphinxcode{\sphinxupquote{utils.ac\_utils.}}\sphinxbfcode{\sphinxupquote{decToBinConversion}}}{\emph{\DUrole{n}{no}\DUrole{p}{:}\DUrole{w}{  }\DUrole{n}{float}}, \emph{\DUrole{n}{precision}\DUrole{p}{:}\DUrole{w}{  }\DUrole{n}{int}}}{{ $\rightarrow$ str}}
\pysigstopsignatures
\sphinxAtStartPar
Converts a decimal number to binary accepts fraction as well
\begin{description}
\sphinxlineitem{Parameters:}\begin{description}
\sphinxlineitem{no: float }
\sphinxAtStartPar
decimal number can consist fraction part as well

\sphinxlineitem{precision: int}
\sphinxAtStartPar
precision required for the fractinal part returns fractional part to that precision level

\end{description}

\sphinxlineitem{Returns:}\begin{description}
\sphinxlineitem{binary: str}
\sphinxAtStartPar
returns the binary conversion of a decimal number with user\sphinxhyphen{}defined precision

\end{description}

\end{description}

\end{fulllineitems}

\index{getBinaryFractionValue() (in module utils.ac\_utils)@\spxentry{getBinaryFractionValue()}\spxextra{in module utils.ac\_utils}}

\begin{fulllineitems}
\phantomsection\label{\detokenize{utils:utils.ac_utils.getBinaryFractionValue}}
\pysigstartsignatures
\pysiglinewithargsret{\sphinxcode{\sphinxupquote{utils.ac\_utils.}}\sphinxbfcode{\sphinxupquote{getBinaryFractionValue}}}{\emph{\DUrole{n}{binaryFraction}}}{}
\pysigstopsignatures
\sphinxAtStartPar
Compute the binary fraction value using the formula of:
(2\textasciicircum{}\sphinxhyphen{}1) * 1st bit + (2\textasciicircum{}\sphinxhyphen{}2) * 2nd bit + …
\begin{description}
\sphinxlineitem{Parameters:}\begin{description}
\sphinxlineitem{binaryFraction: str}
\sphinxAtStartPar
binary string of the fractional part.

\end{description}

\sphinxlineitem{Returns:}\begin{description}
\sphinxlineitem{value: float}
\sphinxAtStartPar
returns the fractional part in decimal

\end{description}

\end{description}

\end{fulllineitems}



\subsection{utils.bit\_array\_utils module}
\label{\detokenize{utils:module-utils.bit_array_utils}}\label{\detokenize{utils:utils-bit-array-utils-module}}\index{module@\spxentry{module}!utils.bit\_array\_utils@\spxentry{utils.bit\_array\_utils}}\index{utils.bit\_array\_utils@\spxentry{utils.bit\_array\_utils}!module@\spxentry{module}}\index{bitarray\_to\_int() (in module utils.bit\_array\_utils)@\spxentry{bitarray\_to\_int()}\spxextra{in module utils.bit\_array\_utils}}

\begin{fulllineitems}
\phantomsection\label{\detokenize{utils:utils.bit_array_utils.bitarray_to_int}}
\pysigstartsignatures
\pysiglinewithargsret{\sphinxcode{\sphinxupquote{utils.bit\_array\_utils.}}\sphinxbfcode{\sphinxupquote{bitarray\_to\_int}}}{\emph{\DUrole{n}{bit\_array}\DUrole{p}{:}\DUrole{w}{  }\DUrole{n}{bitarray.bitarray}}}{}
\pysigstopsignatures
\sphinxAtStartPar
Converts bitarray to int.
\begin{description}
\sphinxlineitem{Parameters:}\begin{description}
\sphinxlineitem{bit\_array: BITARRAY}
\sphinxAtStartPar
bits to be converted

\end{description}

\sphinxlineitem{Returns:}\begin{description}
\sphinxlineitem{dec: int}
\sphinxAtStartPar
decimal\_equivalent of bit\_array

\end{description}

\end{description}

\end{fulllineitems}

\index{bitarrays\_to\_float() (in module utils.bit\_array\_utils)@\spxentry{bitarrays\_to\_float()}\spxextra{in module utils.bit\_array\_utils}}

\begin{fulllineitems}
\phantomsection\label{\detokenize{utils:utils.bit_array_utils.bitarrays_to_float}}
\pysigstartsignatures
\pysiglinewithargsret{\sphinxcode{\sphinxupquote{utils.bit\_array\_utils.}}\sphinxbfcode{\sphinxupquote{bitarrays\_to\_float}}}{\emph{\DUrole{n}{uint\_x\_bitarray}\DUrole{p}{:}\DUrole{w}{  }\DUrole{n}{bitarray.bitarray}}, \emph{\DUrole{n}{frac\_x\_bitarray}\DUrole{p}{:}\DUrole{w}{  }\DUrole{n}{bitarray.bitarray}}}{{ $\rightarrow$ float}}
\pysigstopsignatures
\sphinxAtStartPar
Converts bitarrays corresponding to integer and fractional part of a floatating point number to a float.
\begin{description}
\sphinxlineitem{Parameters:}\begin{description}
\sphinxlineitem{uint\_x\_bitarray: BitArray }
\sphinxAtStartPar
bitarray corresponding to the integer part of x

\sphinxlineitem{frac\_x\_bitarray: BitArray }
\sphinxAtStartPar
bitarray corresponding to the fractional part of x

\end{description}

\sphinxlineitem{Returns:}
\sphinxAtStartPar
x: float, the floating point number

\end{description}

\end{fulllineitems}

\index{float\_to\_bitarrays() (in module utils.bit\_array\_utils)@\spxentry{float\_to\_bitarrays()}\spxextra{in module utils.bit\_array\_utils}}

\begin{fulllineitems}
\phantomsection\label{\detokenize{utils:utils.bit_array_utils.float_to_bitarrays}}
\pysigstartsignatures
\pysiglinewithargsret{\sphinxcode{\sphinxupquote{utils.bit\_array\_utils.}}\sphinxbfcode{\sphinxupquote{float\_to\_bitarrays}}}{\emph{\DUrole{n}{x}\DUrole{p}{:}\DUrole{w}{  }\DUrole{n}{float}}, \emph{\DUrole{n}{max\_precision}\DUrole{p}{:}\DUrole{w}{  }\DUrole{n}{int}}}{{ $\rightarrow$ Tuple\DUrole{p}{{[}}bitarray.bitarray\DUrole{p}{,}\DUrole{w}{  }bitarray.bitarray\DUrole{p}{{]}}}}
\pysigstopsignatures
\sphinxAtStartPar
Convert floating point number to binary with the given max\_precision
Utility function to obtain binary representation of the floating point number.
We return a tuple of binary representations of the integer part and the fraction part of the
floating point number.
\begin{description}
\sphinxlineitem{Parameters:}\begin{description}
\sphinxlineitem{x: float}
\sphinxAtStartPar
input floating point number

\sphinxlineitem{max\_precision: int }
\sphinxAtStartPar
max binary precision (after the decimal point) to which we should return the bitarray

\end{description}

\sphinxlineitem{Returns:}
\sphinxAtStartPar
Tuple{[}BitArray, BitArray{]}: returns (uint\_x\_bitarray, frac\_x\_bitarray)

\end{description}

\end{fulllineitems}

\index{int\_to\_bitarray() (in module utils.bit\_array\_utils)@\spxentry{int\_to\_bitarray()}\spxextra{in module utils.bit\_array\_utils}}

\begin{fulllineitems}
\phantomsection\label{\detokenize{utils:utils.bit_array_utils.int_to_bitarray}}
\pysigstartsignatures
\pysiglinewithargsret{\sphinxcode{\sphinxupquote{utils.bit\_array\_utils.}}\sphinxbfcode{\sphinxupquote{int\_to\_bitarray}}}{\emph{\DUrole{n}{x}\DUrole{p}{:}\DUrole{w}{  }\DUrole{n}{int}}, \emph{\DUrole{n}{bit\_width}\DUrole{o}{=}\DUrole{default_value}{None}}}{{ $\rightarrow$ bitarray.bitarray}}
\pysigstopsignatures
\sphinxAtStartPar
Converts int to bits.
\begin{description}
\sphinxlineitem{Parameters:}\begin{description}
\sphinxlineitem{x: int}
\sphinxAtStartPar
integer to be converted

\sphinxlineitem{bit\_width: int, default = None}
\sphinxAtStartPar
length

\end{description}

\sphinxlineitem{Returns:}\begin{description}
\sphinxlineitem{bit: BITARRAY}
\sphinxAtStartPar
binary equivalent of x

\end{description}

\end{description}

\end{fulllineitems}



\subsection{utils.file\_utils module}
\label{\detokenize{utils:module-utils.file_utils}}\label{\detokenize{utils:utils-file-utils-module}}\index{module@\spxentry{module}!utils.file\_utils@\spxentry{utils.file\_utils}}\index{utils.file\_utils@\spxentry{utils.file\_utils}!module@\spxentry{module}}\index{convert\_bytes() (in module utils.file\_utils)@\spxentry{convert\_bytes()}\spxextra{in module utils.file\_utils}}

\begin{fulllineitems}
\phantomsection\label{\detokenize{utils:utils.file_utils.convert_bytes}}
\pysigstartsignatures
\pysiglinewithargsret{\sphinxcode{\sphinxupquote{utils.file\_utils.}}\sphinxbfcode{\sphinxupquote{convert\_bytes}}}{\emph{\DUrole{n}{num}}}{{ $\rightarrow$ str}}
\pysigstopsignatures
\sphinxAtStartPar
This function will convert bytes to MB…. GB… etc.
\begin{description}
\sphinxlineitem{Parameters:}\begin{description}
\sphinxlineitem{x: float}
\sphinxAtStartPar
input floating point number

\sphinxlineitem{max\_precision: int}
\sphinxAtStartPar
max binary precision (after the decimal point) to which we should return the bitarray

\end{description}

\sphinxlineitem{Returns:}\begin{description}
\sphinxlineitem{file\_size: str}
\sphinxAtStartPar
file size to the nearest MB…. GB…

\end{description}

\end{description}

\begin{sphinxVerbatim}[commandchars=\\\{\}]
\PYG{g+gp}{\PYGZgt{}\PYGZgt{}\PYGZgt{} }\PYG{n}{convert\PYGZus{}bytes}\PYG{p}{(}\PYG{l+m+mi}{10580}\PYG{p}{)}
\PYG{g+go}{    10.3 KB}
\end{sphinxVerbatim}

\end{fulllineitems}

\index{epoch\_to\_datetime() (in module utils.file\_utils)@\spxentry{epoch\_to\_datetime()}\spxextra{in module utils.file\_utils}}

\begin{fulllineitems}
\phantomsection\label{\detokenize{utils:utils.file_utils.epoch_to_datetime}}
\pysigstartsignatures
\pysiglinewithargsret{\sphinxcode{\sphinxupquote{utils.file\_utils.}}\sphinxbfcode{\sphinxupquote{epoch\_to\_datetime}}}{\emph{\DUrole{n}{epoch\_time}}}{}
\pysigstopsignatures
\sphinxAtStartPar
This function converts epoch into datetime
\begin{description}
\sphinxlineitem{Parameters:}\begin{description}
\sphinxlineitem{epoch\_time: int}
\sphinxAtStartPar
time in epoch scale

\end{description}

\sphinxlineitem{Returns:}\begin{description}
\sphinxlineitem{date: datetime}
\sphinxAtStartPar
standard date and time of the epoch time.

\end{description}

\end{description}

\begin{sphinxVerbatim}[commandchars=\\\{\}]
\PYG{g+gp}{\PYGZgt{}\PYGZgt{}\PYGZgt{} }\PYG{n}{epoch\PYGZus{}to\PYGZus{}datetime}\PYG{p}{(}\PYG{l+m+mi}{12452687}\PYG{p}{)}
\PYG{g+go}{    1970\PYGZhy{}05\PYGZhy{}25 08:34:47}
\end{sphinxVerbatim}

\end{fulllineitems}

\index{file\_creation() (in module utils.file\_utils)@\spxentry{file\_creation()}\spxextra{in module utils.file\_utils}}

\begin{fulllineitems}
\phantomsection\label{\detokenize{utils:utils.file_utils.file_creation}}
\pysigstartsignatures
\pysiglinewithargsret{\sphinxcode{\sphinxupquote{utils.file\_utils.}}\sphinxbfcode{\sphinxupquote{file\_creation}}}{\emph{\DUrole{n}{file\_path}}}{}
\pysigstopsignatures
\sphinxAtStartPar
This function will return the file creation date
\begin{description}
\sphinxlineitem{Parameters:}\begin{description}
\sphinxlineitem{file\_path: str}
\sphinxAtStartPar
path of the file

\end{description}

\sphinxlineitem{Returns:}\begin{description}
\sphinxlineitem{file\_creation: Datetime}
\sphinxAtStartPar
file creation time in standard format

\end{description}

\end{description}

\begin{sphinxVerbatim}[commandchars=\\\{\}]
\PYG{g+gp}{\PYGZgt{}\PYGZgt{}\PYGZgt{} }\PYG{n}{file\PYGZus{}creation}\PYG{p}{(}\PYG{l+s+s1}{\PYGZsq{}}\PYG{l+s+s1}{/Users/jenish/Library/CloudStorage/Dropbox/crypto/ANS/code/ANS/utils/utils.py}\PYG{l+s+s1}{\PYGZsq{}}\PYG{p}{)}
\PYG{g+go}{    2023\PYGZhy{}03\PYGZhy{}13 23:11:04.054830}
\end{sphinxVerbatim}

\end{fulllineitems}

\index{file\_last\_modified() (in module utils.file\_utils)@\spxentry{file\_last\_modified()}\spxextra{in module utils.file\_utils}}

\begin{fulllineitems}
\phantomsection\label{\detokenize{utils:utils.file_utils.file_last_modified}}
\pysigstartsignatures
\pysiglinewithargsret{\sphinxcode{\sphinxupquote{utils.file\_utils.}}\sphinxbfcode{\sphinxupquote{file\_last\_modified}}}{\emph{\DUrole{n}{file\_path}}}{}
\pysigstopsignatures
\sphinxAtStartPar
This function will return the file modified datetime
\begin{description}
\sphinxlineitem{Parameters:}\begin{description}
\sphinxlineitem{file\_path: str}
\sphinxAtStartPar
path of the file

\end{description}

\sphinxlineitem{Returns:}\begin{description}
\sphinxlineitem{file\_moodified\_time: Datetime}
\sphinxAtStartPar
last modified time in standard format

\end{description}

\end{description}

\begin{sphinxVerbatim}[commandchars=\\\{\}]
\PYG{g+gp}{\PYGZgt{}\PYGZgt{}\PYGZgt{} }\PYG{n}{file\PYGZus{}last\PYGZus{}modified}\PYG{p}{(}\PYG{l+s+s1}{\PYGZsq{}}\PYG{l+s+s1}{/Users/jenish/Library/CloudStorage/Dropbox/crypto/ANS/code/ANS/utils/utils.py}\PYG{l+s+s1}{\PYGZsq{}}\PYG{p}{)}
\PYG{g+go}{    2023\PYGZhy{}03\PYGZhy{}13 23:11:01.540858}
\end{sphinxVerbatim}

\end{fulllineitems}

\index{file\_name() (in module utils.file\_utils)@\spxentry{file\_name()}\spxextra{in module utils.file\_utils}}

\begin{fulllineitems}
\phantomsection\label{\detokenize{utils:utils.file_utils.file_name}}
\pysigstartsignatures
\pysiglinewithargsret{\sphinxcode{\sphinxupquote{utils.file\_utils.}}\sphinxbfcode{\sphinxupquote{file\_name}}}{\emph{\DUrole{n}{file\_path}}}{}
\pysigstopsignatures
\sphinxAtStartPar
This function will return the file name.
\begin{description}
\sphinxlineitem{Parameters:}\begin{description}
\sphinxlineitem{file\_path: str}
\sphinxAtStartPar
path of the file

\end{description}

\sphinxlineitem{Returns:}\begin{description}
\sphinxlineitem{file\_name: str}
\sphinxAtStartPar
name of the file with extension

\end{description}

\end{description}

\begin{sphinxVerbatim}[commandchars=\\\{\}]
\PYG{g+gp}{\PYGZgt{}\PYGZgt{}\PYGZgt{} }\PYG{n}{file\PYGZus{}name}\PYG{p}{(}\PYG{l+s+s1}{\PYGZsq{}}\PYG{l+s+s1}{/Users/jenish/Library/CloudStorage/Dropbox/crypto/ANS/code/ANS/utils/utils.py}\PYG{l+s+s1}{\PYGZsq{}}\PYG{p}{)}
\PYG{g+go}{    utils.py}
\end{sphinxVerbatim}

\end{fulllineitems}

\index{file\_size() (in module utils.file\_utils)@\spxentry{file\_size()}\spxextra{in module utils.file\_utils}}

\begin{fulllineitems}
\phantomsection\label{\detokenize{utils:utils.file_utils.file_size}}
\pysigstartsignatures
\pysiglinewithargsret{\sphinxcode{\sphinxupquote{utils.file\_utils.}}\sphinxbfcode{\sphinxupquote{file\_size}}}{\emph{\DUrole{n}{file\_path}}}{}
\pysigstopsignatures
\sphinxAtStartPar
This function will return the file size.
\begin{description}
\sphinxlineitem{Parameters:}\begin{description}
\sphinxlineitem{file\_path: str}
\sphinxAtStartPar
path of the file

\end{description}

\sphinxlineitem{Returns:}\begin{description}
\sphinxlineitem{file\_size: int}
\sphinxAtStartPar
size of file in bits

\end{description}

\end{description}

\begin{sphinxVerbatim}[commandchars=\\\{\}]
\PYG{g+gp}{\PYGZgt{}\PYGZgt{}\PYGZgt{} }\PYG{n}{file\PYGZus{}size}\PYG{p}{(}\PYG{l+s+s1}{\PYGZsq{}}\PYG{l+s+s1}{/Users/jenish/Library/CloudStorage/Dropbox/crypto/ANS/code/ANS/utils/utils.py}\PYG{l+s+s1}{\PYGZsq{}}\PYG{p}{)}
\PYG{g+go}{    3305}
\end{sphinxVerbatim}

\end{fulllineitems}

\index{file\_summary() (in module utils.file\_utils)@\spxentry{file\_summary()}\spxextra{in module utils.file\_utils}}

\begin{fulllineitems}
\phantomsection\label{\detokenize{utils:utils.file_utils.file_summary}}
\pysigstartsignatures
\pysiglinewithargsret{\sphinxcode{\sphinxupquote{utils.file\_utils.}}\sphinxbfcode{\sphinxupquote{file\_summary}}}{\emph{\DUrole{n}{file\_path}}}{}
\pysigstopsignatures
\sphinxAtStartPar
Compautes the summary of the file. Runs the file util functions.
\begin{description}
\sphinxlineitem{Parameters:}\begin{description}
\sphinxlineitem{file\_path: str}
\sphinxAtStartPar
path of the file

\end{description}

\sphinxlineitem{Returns:}
\sphinxAtStartPar
name, size, creation, modification: Tuple{[}str, str, datetime, datetime{]}

\end{description}

\begin{sphinxVerbatim}[commandchars=\\\{\}]
\PYG{g+gp}{\PYGZgt{}\PYGZgt{}\PYGZgt{} }\PYG{n}{file\PYGZus{}summary}\PYG{p}{(}\PYG{l+s+s1}{\PYGZsq{}}\PYG{l+s+s1}{/Users/jenish/Library/CloudStorage/Dropbox/crypto/ANS/code/ANS/utils/utils.py}\PYG{l+s+s1}{\PYGZsq{}}\PYG{p}{)}
\PYG{g+go}{    (\PYGZsq{}utils.py\PYGZsq{}, 3305, datetime.datetime(2023, 3, 13, 23, 11, 4, 54830), datetime.datetime(2023, 3, 13, 23, 11, 1, 540858))}
\end{sphinxVerbatim}

\end{fulllineitems}



\subsection{utils.utils module}
\label{\detokenize{utils:module-utils.utils}}\label{\detokenize{utils:utils-utils-module}}\index{module@\spxentry{module}!utils.utils@\spxentry{utils.utils}}\index{utils.utils@\spxentry{utils.utils}!module@\spxentry{module}}\index{convert\_list\_to\_string() (in module utils.utils)@\spxentry{convert\_list\_to\_string()}\spxextra{in module utils.utils}}

\begin{fulllineitems}
\phantomsection\label{\detokenize{utils:utils.utils.convert_list_to_string}}
\pysigstartsignatures
\pysiglinewithargsret{\sphinxcode{\sphinxupquote{utils.utils.}}\sphinxbfcode{\sphinxupquote{convert\_list\_to\_string}}}{\emph{\DUrole{n}{l}\DUrole{p}{:}\DUrole{w}{  }\DUrole{n}{list}}}{{ $\rightarrow$ str}}
\pysigstopsignatures
\end{fulllineitems}

\index{encode\_symbols\_to\_integer() (in module utils.utils)@\spxentry{encode\_symbols\_to\_integer()}\spxextra{in module utils.utils}}

\begin{fulllineitems}
\phantomsection\label{\detokenize{utils:utils.utils.encode_symbols_to_integer}}
\pysigstartsignatures
\pysiglinewithargsret{\sphinxcode{\sphinxupquote{utils.utils.}}\sphinxbfcode{\sphinxupquote{encode\_symbols\_to\_integer}}}{\emph{\DUrole{n}{symbols}\DUrole{p}{:}\DUrole{w}{  }\DUrole{n}{list}}}{}
\pysigstopsignatures
\sphinxAtStartPar
Encodes each symbol to a integer starting from 0. Helper function for encoding\_table
\begin{description}
\sphinxlineitem{Parameters:}\begin{description}
\sphinxlineitem{symbols: list}
\sphinxAtStartPar
list of symbols to be encoded

\end{description}

\sphinxlineitem{Returns:}\begin{description}
\sphinxlineitem{encoded\_list: dict }
\sphinxAtStartPar
dict with the encoded values as keys

\end{description}

\end{description}

\end{fulllineitems}

\index{encoding\_table() (in module utils.utils)@\spxentry{encoding\_table()}\spxextra{in module utils.utils}}

\begin{fulllineitems}
\phantomsection\label{\detokenize{utils:utils.utils.encoding_table}}
\pysigstartsignatures
\pysiglinewithargsret{\sphinxcode{\sphinxupquote{utils.utils.}}\sphinxbfcode{\sphinxupquote{encoding\_table}}}{\emph{\DUrole{n}{table\_elements}}}{}
\pysigstopsignatures
\sphinxAtStartPar
Creates an encoding table given table elements for ANS. This table can be used to determing the symbol spread function. Helper function for ANS.rANS.encoding\_table().
\begin{description}
\sphinxlineitem{Paramaters:}\begin{description}
\sphinxlineitem{table\_elements: Tuple(symbols, x\_prev, x\_new)}
\sphinxAtStartPar
table elements is a list of symbol, x\_prev, x\_new
encoding\_table: matrix (A) of size len(symbol) * max(x\_new) where A\_\{symbol,x\_new\} = x\_prev

\end{description}

\sphinxlineitem{Returns: }\begin{description}
\sphinxlineitem{table: pd.Dataframe}
\sphinxAtStartPar
the encoding table. Usually for ANS.

\end{description}

\end{description}

\end{fulllineitems}

\index{get\_symbols() (in module utils.utils)@\spxentry{get\_symbols()}\spxextra{in module utils.utils}}

\begin{fulllineitems}
\phantomsection\label{\detokenize{utils:utils.utils.get_symbols}}
\pysigstartsignatures
\pysiglinewithargsret{\sphinxcode{\sphinxupquote{utils.utils.}}\sphinxbfcode{\sphinxupquote{get\_symbols}}}{\emph{\DUrole{n}{symbols}\DUrole{p}{:}\DUrole{w}{  }\DUrole{n}{list}}, \emph{\DUrole{n}{frequency}\DUrole{p}{:}\DUrole{w}{  }\DUrole{n}{list}}, \emph{\DUrole{n}{no\_symbols}\DUrole{p}{:}\DUrole{w}{  }\DUrole{n}{int}}}{{ $\rightarrow$ list}}
\pysigstopsignatures
\sphinxAtStartPar
Get arbitrary number of symbols following a particular distribution. Uses inversion sampling to sampling symbols. Used to test compressors.
\begin{description}
\sphinxlineitem{Parameters:}\begin{description}
\sphinxlineitem{symbols: list}
\sphinxAtStartPar
list of symbols

\sphinxlineitem{frequency: list}
\sphinxAtStartPar
list of freqency associated with a particular symbol

\sphinxlineitem{no\_symbols: int}
\sphinxAtStartPar
number of symbols you want to sample

\end{description}

\sphinxlineitem{Returns:}\begin{description}
\sphinxlineitem{symbols: list}
\sphinxAtStartPar
list of symbols following a particular distribution

\end{description}

\end{description}

\begin{sphinxVerbatim}[commandchars=\\\{\}]
\PYG{g+gp}{\PYGZgt{}\PYGZgt{}\PYGZgt{} }\PYG{n}{get\PYGZus{}symbols}\PYG{p}{(}\PYG{p}{[}\PYG{l+s+s1}{\PYGZsq{}}\PYG{l+s+s1}{a}\PYG{l+s+s1}{\PYGZsq{}}\PYG{p}{,} \PYG{l+s+s1}{\PYGZsq{}}\PYG{l+s+s1}{b}\PYG{l+s+s1}{\PYGZsq{}}\PYG{p}{,} \PYG{l+s+s1}{\PYGZsq{}}\PYG{l+s+s1}{c}\PYG{l+s+s1}{\PYGZsq{}}\PYG{p}{,} \PYG{l+s+s1}{\PYGZsq{}}\PYG{l+s+s1}{d}\PYG{l+s+s1}{\PYGZsq{}}\PYG{p}{,} \PYG{l+s+s1}{\PYGZsq{}}\PYG{l+s+s1}{z}\PYG{l+s+s1}{\PYGZsq{}}\PYG{p}{]}\PYG{p}{,} \PYG{p}{[}\PYG{l+m+mf}{0.2}\PYG{p}{,}\PYG{l+m+mf}{0.3}\PYG{p}{,}\PYG{l+m+mf}{0.1}\PYG{p}{,}\PYG{l+m+mf}{0.1}\PYG{p}{,}\PYG{l+m+mf}{0.4}\PYG{p}{]}\PYG{p}{,} \PYG{l+m+mi}{10}\PYG{p}{)}
\PYG{g+go}{    [\PYGZsq{}z\PYGZsq{}, \PYGZsq{}a\PYGZsq{}, \PYGZsq{}b\PYGZsq{}, \PYGZsq{}b\PYGZsq{}, \PYGZsq{}z\PYGZsq{}, \PYGZsq{}z\PYGZsq{}, \PYGZsq{}b\PYGZsq{}, \PYGZsq{}b\PYGZsq{}, \PYGZsq{}b\PYGZsq{}, \PYGZsq{}a\PYGZsq{}]}
\end{sphinxVerbatim}

\end{fulllineitems}



\subsection{Module contents}
\label{\detokenize{utils:module-utils}}\label{\detokenize{utils:module-contents}}\index{module@\spxentry{module}!utils@\spxentry{utils}}\index{utils@\spxentry{utils}!module@\spxentry{module}}

\chapter{Indices and tables}
\label{\detokenize{index:indices-and-tables}}\begin{itemize}
\item {} 
\sphinxAtStartPar
\DUrole{xref,std,std-ref}{genindex}

\item {} 
\sphinxAtStartPar
\DUrole{xref,std,std-ref}{modindex}

\item {} 
\sphinxAtStartPar
\DUrole{xref,std,std-ref}{search}

\end{itemize}


\renewcommand{\indexname}{Python Module Index}
\begin{sphinxtheindex}
\let\bigletter\sphinxstyleindexlettergroup
\bigletter{a}
\item\relax\sphinxstyleindexentry{ANS}\sphinxstyleindexpageref{ANS:\detokenize{module-ANS}}
\item\relax\sphinxstyleindexentry{arithmetic\_coding}\sphinxstyleindexpageref{arithmetic_coding:\detokenize{module-arithmetic_coding}}
\indexspace
\bigletter{c}
\item\relax\sphinxstyleindexentry{core}\sphinxstyleindexpageref{core:\detokenize{module-core}}
\item\relax\sphinxstyleindexentry{core.data}\sphinxstyleindexpageref{core:\detokenize{module-core.data}}
\indexspace
\bigletter{f}
\item\relax\sphinxstyleindexentry{file\_compressor}\sphinxstyleindexpageref{file_compressor:\detokenize{module-file_compressor}}
\indexspace
\bigletter{h}
\item\relax\sphinxstyleindexentry{huffman}\sphinxstyleindexpageref{huffman:\detokenize{module-huffman}}
\indexspace
\bigletter{s}
\item\relax\sphinxstyleindexentry{sANS}\sphinxstyleindexpageref{sANS:\detokenize{module-sANS}}
\item\relax\sphinxstyleindexentry{symmetric\_numeral}\sphinxstyleindexpageref{symmetric_numeral:\detokenize{module-symmetric_numeral}}
\indexspace
\bigletter{u}
\item\relax\sphinxstyleindexentry{uABS}\sphinxstyleindexpageref{uABS:\detokenize{module-uABS}}
\item\relax\sphinxstyleindexentry{utils}\sphinxstyleindexpageref{utils:\detokenize{module-utils}}
\item\relax\sphinxstyleindexentry{utils.ac\_utils}\sphinxstyleindexpageref{utils:\detokenize{module-utils.ac_utils}}
\item\relax\sphinxstyleindexentry{utils.bit\_array\_utils}\sphinxstyleindexpageref{utils:\detokenize{module-utils.bit_array_utils}}
\item\relax\sphinxstyleindexentry{utils.file\_utils}\sphinxstyleindexpageref{utils:\detokenize{module-utils.file_utils}}
\item\relax\sphinxstyleindexentry{utils.utils}\sphinxstyleindexpageref{utils:\detokenize{module-utils.utils}}
\end{sphinxtheindex}

\renewcommand{\indexname}{Index}
\printindex
\end{document}